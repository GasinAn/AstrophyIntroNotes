\chapter{主序和主序后恒星演化\\Main Sequence and Post-Main-Sequence Stellar Evolution}

原恒星演化(protostellar evolution), $1M_\odot$:
\begin{enumerate}
    \item 自由下落塌缩(free-fall collapse). 光学薄, 等温塌缩(isothermal collapse). 中心密度稍大, 塌缩速率更大, 密度差更大, 塌缩速率差更大\dots, 自内向外塌缩(inside-out collapse).
    \item 核心光学厚, 绝热塌缩(adiabatic collapse), 塌缩速率减慢, 核心接近流体静力学平衡, 核心形成原恒星. 核外自由下落, 产生激波\footnote{因为下落速度大于声速.}, 加热核.
    \item 尘埃蒸发, 核不透明度下降.
    \item $\text{H}_2$瓦解, 吸收能量, 核再次塌缩.
    \item 核重新流体静力学平衡, 持续吸积, 至$\text{D}$点燃.
    \item $\text{D}$燃尽, 光度剧降且温度缓降, 进入主序前.
\end{enumerate}

主序前演化(pre-main-sequence evolution):
\begin{itemize}
    \item $1M_\odot$:
    \begin{enumerate}
        \item 表面$\text{H}^-$高不透明度$\Longrightarrow$完全对流. 星体塌缩.
        \item 核温度升高$\Longrightarrow$电离$\Longrightarrow$不透明度下降$\Longrightarrow$辐射核. 星体塌缩.
        \item 核反应(pp和CNO)开始.
        \item CNO温度依赖$\Longrightarrow$高产能率$\Longrightarrow$高温度梯度$\Longrightarrow$对流核.  高产能率$\Longrightarrow$核膨胀$\Longrightarrow$温度下降且光度下降.
        \item $\text{C}$燃尽, 恒星稳定, 进入主序.
    \end{enumerate}
    \item $<0.5M_\odot$, 无$\text{C}$燃烧.
    \item $<0.072M_\odot$, 无$\text{H}$燃烧, 不可持续, 失败.
\end{itemize}

褐矮星(brown dwarf): $>0.06M_\odot$, $\text{Li}$燃烧; $>0.013M_\odot$, $\text{D}$燃烧. 光谱型L和T.

零龄主序(zero-age main sequence, ZAMS): 刚到主序, 开始稳定$\text{H}$燃烧. $>1.2M_\odot$, CNO, 对流核\footnote{这能使得核内成分保持均匀.}; $0.3M_\odot\sim1.2M_\odot$, pp, 辐射核; $<0.3M_\odot$\footnote{书上只说``最低质量'', 我猜是$<0.3M_\odot$都行.}, pp, 对流核.

小质量主序演化(low-mass main-sequence evolution), $1M_\odot$: 光度, 半径, 温度稳定增加, $\text{H}\to\text{He}$, 平均分子量增加, 压强不足, 核塌缩, 反应率增加.

对流超射(convective overshooting): 对流泡泡因惯性冲出对流区.

中质量主序演化(intermediate-mass main-sequence evolution), $5M_\odot$: 核稍稍损失质量, 形成小成分梯度, 末期星体收缩, 光度和温度增加.

$>10M_\odot$, 对流核消失.

主序后演化(post-main-sequence evolution): $\text{H}$燃烧止, 亚巨星支(subgiant branch, SGB), 红巨星支(red giant branch, RGB), 红巨星支上端\footnote{很火的TRGB!}(red giant tip), 蓝向/红向\footnote{这里的``蓝向/红向''是我计几取的名儿\dots}(blueward/redward)水平支(horizontal branch, HB), 早期/热脉冲渐近巨星支(early/thermal-pulse asymptotic giant branch, E/TP-AGB), 渐近巨星后支(post-asymptotic giant branch, Post-AGB).

挖掘(dredge-up). 超星风(superwind).

$1M_\odot$:
\begin{enumerate}
    \item 核收缩, 厚$\text{H}$燃烧壳, 光度增加, 半径稍增, 温度下降.
    \item 亚巨星支, 核迅速收缩, 星体膨胀, 温度下降, 光度增加.
    \item 温度下降$\Longrightarrow$ $\text{H}^-$增加$\Longrightarrow$光球不透明度增加$\Longrightarrow$表面对流.
    \item 对流层深入中心, 红巨星支, 光度增加, 第一次挖掘. $\text{Li}$燃尽, 光球化学成分变化, 红巨星支上端.
    \item 核收缩, 强电子简并, 压强不依赖于温度. 3$\mathrm{\alpha}$反应开始, 温度增加, 压强不变, 温度不下降,  3$\mathrm{\alpha}$反应高温度依赖, 氦闪(Helium core flash).
    \item 3$\mathrm{\alpha}$反应, $\text{He}$燃烧, 核膨胀, $\text{H}$燃烧壳膨胀(被外推), 产能率下降, 光度下降, 壳收缩, 温度升高, 蓝向水平支. 壳收缩最终导致产能率再升高.
    \item 核平均分子量增加, 至核收缩, 壳膨胀冷却, 红向水平支.
    \item $\text{He}$壳燃烧, 早期渐近巨星支, 壳膨胀, 温度下降, 对流层二次深入中心(原因同第一次挖掘), 第二次挖掘.
    \item $\text{He}$壳燃烧初止, 热脉冲渐近巨星支, $\text{H}$燃烧层生成$\text{He}$落入$\text{He}$层, $\text{He}$层部分简并, 氦壳闪(Helium core flash), $\text{H}$燃烧层外推, 冷却停止燃烧, 氦壳闪终止, $\text{H}$燃烧层恢复, 循环. 光度增加, 温度降低.
    \item 半径大且脉动, 引力束缚弱, 光度高, 辐射压大, 快速质量损失, 末期超星风, 渐近巨星后支.
\end{enumerate}

$5M_\odot$:
\begin{enumerate}
    \item 整星KH时标收缩, 光度稍增, 半径下降, 温度升高.
    \item 厚$\text{H}$燃烧壳, 壳稍膨胀, 光度下降, 温度下降.
    \item 亚巨星支, 核迅速收缩, 星体膨胀, 温度下降, 光度增加但最后稍降.
    \item 温度下降$\Longrightarrow$ $\text{H}^-$增加$\Longrightarrow$光球不透明度增加$\Longrightarrow$表面对流.
    \item 对流层深入中心, 红巨星支, 光度增加, 第一次挖掘. $\text{Li}$燃尽, 光球化学成分变化, 红巨星支上端.
    \item 3$\mathrm{\alpha}$反应, $\text{He}$燃烧, 核膨胀, $\text{H}$燃烧壳膨胀(被外推), 产能率下降, 光度下降, 壳收缩, 温度升高, 蓝向水平支.  壳收缩最终导致产能率再升高.
    \item 核平均分子量增加, 至核收缩, 壳膨胀冷却, 红向水平支.
    \item $\text{He}$壳燃烧, 早期渐近巨星支, 壳膨胀, 温度下降, 对流层二次深入中心(原因同第一次挖掘), 第二次挖掘.
    \item $\text{He}$壳燃烧初止, 热脉冲渐近巨星支, 氦壳闪. 光度增加, 温度降低.
    \item 对流层三次深入中心, 第三次挖掘.
    \item 半径大且脉动, 引力束缚弱, 光度高, 辐射压大, 快速质量损失, 末期超星风, 渐近巨星后支.
\end{enumerate}

脉泽(maser): 微波受激辐射.

星族(stellar population): 年龄, 化学组成, 空间分布和运动特性. 星族I: 年轻, 富金属, 银盘, 轨道倾角小. 星族II: 年老, 贫金属, 银晕, 轨道倾角大. 星族III: 第一代, 纯洁无瑕.

球状星团(globular cluster), $n\times10^3\uparrow$, 星族II. 疏散星团(open cluster), $n\times10^3\downarrow$, 星族I.

赫氏空隙(Hertzsprung gap): 星团赫罗图空隙. 原因: 主序后快速演化.
