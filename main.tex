% 编译方式: xelatex*2
\documentclass{ctexbook}
\usepackage{amsfonts}
\usepackage{amsmath}
\usepackage{amssymb}
\usepackage{hyperref}
\usepackage{syntonly}
%\syntaxonly
\pagestyle{plain}
\makeatletter
\newcommand{\starttoc}{
    \chapter*{\contentsname}
    \@starttoc{toc}
}
\makeatother
\renewcommand{\tableofcontents}{\twocolumn\starttoc\onecolumn}
\hypersetup{
	colorlinks,
	linkcolor=blue,
	filecolor=pink,
	urlcolor=cyan,
	citecolor=red,
}
\def\b{\boldsymbol}
\title{天文物理导论笔记}
\author{GasinAn}
\begin{document}
    \maketitle
    \noindent Copyright \textcopyright~2021 by GasinAn

\ 

\noindent All rights reserved. No part of this book may be reproduced, in any form or by any means, without permission in writing from the publisher, except by a BNUer.

\ 

\noindent The author and publisher of this book have used their best efforts in preparing this book. These efforts include the development, research, and testing of the theories, technologies and programs to determine their effectiveness. The author and publisher make no warranty of any kind, express or implied, with regard to these techniques or programs contained in this book. The author and publisher shall not be liable in any event of incidental or consequential damages in connection with, or arising out of, the furnishing, performance, or use of these techniques or programs.

\ 

\noindent Printed in China

    \tableofcontents
    \chapter{天球\\The Celestial Sphere}

会合周期(synodic period)~$S$.
\begin{equation*}
    1/S = \left\lvert 1/P - 1/P_\oplus \right\rvert.
\end{equation*}
通过定义自己推推就推出来了嘛$\!\text{\~{}}$

赤经(right ascension)~$\alpha$, 赤纬(declination)~$\delta$.
\begin{equation*}
    (\Delta\theta)^2 = (\Delta\alpha\cos\delta)^2+(\Delta\delta)^2
\end{equation*}
记忆方法: 当成勾股定理. 懒得写了, 自己想.

北天极的高度角等于地理纬度, 由此自己推算在天体上中天时刻, 天体赤纬, 天体天顶距和地理纬度的关系\footnote{这是本宝宝留给你们的作业题! 哼!}.

    \chapter{天体力学\\Celestial Mechanics}

看我PPT.

圆锥曲线(conic section)统一方程
\begin{equation*}
    r = \frac{ed}{1+e\cos\theta}.
\end{equation*}
椭圆(ellipse), $ed=a(1-e^2)$. 抛物线(parabola), $ed=2p$. 双曲线(hyperbola), $ed=a(1+e^2)$.

半长轴长(semimajor axis)~$a$, 半短轴长(semimajor axis)~$b$, 离心率(eccentricity)~$e$.

焦点(focal point), 近$\!\text{??}\!$点(perhelion), 远$\!\text{??}\!$点(aphelion).

椭圆面积$A=\pi ab$. 证明: 把单位圆横轴方向拉长$a$倍, 纵轴方向拉长$b$倍.

逃逸速度(escape velocity)~$v_\text{esc}$.
\begin{equation*}
    v_\text{esc} = \sqrt{2GM/r}.
\end{equation*}
第二宇宙速度$11.2\,\text{km/s}$.

质心(center of mass)~$\b{R}$.
\begin{equation*}
    \b{R} = \frac{
        \sum_{i=1}^n m_i\b{r}_i
    }{
        \sum_{i=1}^n m_i
    }.
\end{equation*}

折合质量(reduced mass)~$\mu$.
\begin{equation*}
    \mu = \frac{m_1m_2}{m_1+m_2}.
\end{equation*}
二体问题, 把坐标系建在其中一个天体上, 将其质量强行定为$M=m_1+m_2$, 另一天体质量强行定为$\mu$, 把日地系统的Kepler三定律, 和日地系统中地球的机械能和角动量的表达式中的$M_\odot$都换成$M$, $M_\oplus$都换成$\mu$, 就能得到二体问题的Kepler三定律和两天体的总机械能和角动量.

一些有用的表达式.
\begin{equation*}
    ed = \frac{1}{GM}\frac{L^2}{\mu^2}.
\end{equation*}
\begin{equation*}
    \mathrm{d}A=\frac{1}{2}\frac{L}{\mu}\mathrm{d}t
\end{equation*}
\begin{equation*}
    k=\frac{4\pi^2}{GM}.
\end{equation*}
完蛋了, 把Kepler三定律直接给出来了\dots~第一式推导: 计算perhelion处的$L$, $r$有了, $v$用机械能的表达式算. 第二式推导: 三角形面积是两条边的叉乘的长度的二分之一. 第三式推导: 假装轨道是圆的.

(老师没讲但很重要!)总机械能~$E$.
\begin{equation*}
    E =
    \begin{cases}
        -\frac{GM}{2a} & \text{椭圆}, \\
         0 & \text{抛物}, \\
         \frac{GM}{2a} & \text{双曲}. \\
    \end{cases}
\end{equation*}

维里定理(virial theorem): 系统, 平均总动能$\left\langle T\right\rangle$, 平均总势能$\left\langle V\right\rangle$, 平均总机械能$\left\langle E\right\rangle$,
\begin{equation*}
    2\left\langle T\right\rangle+\left\langle V\right\rangle = 0,
\end{equation*}
\begin{equation*}
    \left\langle E\right\rangle = \frac{1}{2}\left\langle V\right\rangle.
\end{equation*}
推论: 系统稳定, 平均总机械能必小于0.

    \chapter{光的连续谱\\The Continuous Spectrum of Light}

视差(parallax angle): 从天体上看, 地球和太阳的最大角距离.

$1\text{rad}\simeq 206265''$, $1\text{pc}\simeq 206265\text{AU}\simeq 3.26\text{ly}$.

视星等(apparent magnitude)~$m$. 辐射流量(radiant flux)~$F$: 仪器单位面积每秒接收到的能量. 光度(luminosity)~$L$: 天体每秒辐射的总能量.
\begin{equation*}
    m-m_0 = -2.5\lg \frac{F}{F_0}, F=100^{-(m-m_0)/5}F_0.
\end{equation*}
记忆法: 视星等五等, 亮度一百倍; 视星等越小越亮.
\begin{equation*}
    F = \frac{L}{4\pi r^2}.
\end{equation*}

绝对星等(absolute magnitude)~$M$: $10\text{pc}$处视星等. 距离模数(distance modulus)~$m-M$. 作业: 推导距离模数公式.

Stefan--Boltzmann律: $F=\sigma T^4$.~$B(T)=\sigma T^4/\pi$: 黑体垂直于单位面元方向单位立体角内单位时间辐射的能量.
\begin{equation*}
    F = \int_{\theta\in[0,\pi]}B\cos\theta\,\mathrm{d}\Omega.
\end{equation*}
Wien位移\footnote{~``位移''的英文是displacement.}律: $\lambda_\text{max}T\simeq(500\text{nm})(6000\text{K})$.

Planck函数: 
\begin{equation*}
    B_\nu(T)=\frac{2h\nu^3/c^2}{e^{h\nu/kT}-1}.
\end{equation*}
$B_\nu(T)$单位: $\text{J}/(\text{Hz}\cdot\text{s}\cdot\text{m}^2\cdot\text{sr})$. 作业: 推导$B_\lambda(T)$.

Planck函数的证明:
\begin{equation*}
    u_\nu(T)=\frac{4\pi}{c}B_\nu(T).
\end{equation*}
\begin{equation*}
    [u_\nu(T)\mathrm{d}\nu]V=\bar{\epsilon}(\nu)g(\nu)\mathrm{d}\nu.
\end{equation*}
\begin{equation*}
    \bar{\epsilon}(\nu)=\frac{h\nu}{e^{h\nu/kT}-1}=\sum_n (nh\nu)\frac{e^{-nh\nu/kT}}{\sum_n e^{-nh\nu/kT}}.
\end{equation*}
\begin{equation*}
    g(\nu)\mathrm{d}\nu=2\left[4\pi\left(\frac{L\nu}{c}\right)^2\mathrm{d}\left(\frac{L\nu}{c}\right)\right].
\end{equation*}

色指数(color index)~$X-Y$: 不同``波段''视星等的差, 等于不同``波段''绝对星等的差. 热星等(bolometric magnitude)~$m_\text{bol}$和$M_\text{bol}$: 全波段星等. 热改正(bolometric correction)~$BC=m_\text{bol}-V=M_\text{bol}-M_V$.

颜色-颜色图\footnote{不是色色图! 不是!}(color-color diagram): 横轴某色指数, 纵轴另一个. 黑体一条直线, 恒星线在黑体线下.

    \chapter{狭义相对论\\The Theory of Special Relativity}

太简单了, 没啥好记的.

    \chapter{光与物质的相互作用\\The Interaction of Light and Matter}

Kirchhoff律:
\begin{enumerate}
    \item 热致密气体或热固体产生连续谱, 无吸收线.
    \item 热弥漫气体带发射线.
    \item 冷弥漫气体在连续谱源前, 连续谱带吸收线.
\end{enumerate}

补充内容:

刻线间距$d$, 反射光与光栅法线夹角$\theta$, 光谱阶数$n$, 波长$\lambda$, $d\sin\theta=n\lambda$.

波长$\lambda$, 可分辨的最小波长差$\Delta\lambda$, 光谱阶数$n$, 刻线总数$N$, 分辨本领(resolving power)~$R=\lambda/\Delta\lambda=nN$.

Compton效应: 高能光子打低能(静止)电子, 光子波长变长. 逆Compton效应: 低能光子打高能电子, 光子波长变短. $\Delta\lambda=1-\cos\theta$.

$E_n=E_1/n^2$, $r_n=n^2r_1$. $E_1\simeq-13.6\text{eV}$, $r_1\simeq0.05\text{nm}$.

HI量子数$(n,l,m_l,m_s)$. 主量子数$n\in\mathbb{N}_+$, 轨道量子数$l=0,\dots,n-1$, 轨道磁量子数$m_l=-l,\dots,l$, 自旋磁量子数$m_s=-1/2,1/2$~($s=1/2$). 磁场方向为$z$方向, 轨道角动量$L=\sqrt{l(l+1)}\hbar$, $z$方向轨道角动量$L_z=m_l\hbar$, 自旋角动量$S=\sqrt{s(s+1)}\hbar=(\sqrt{3}/2)\hbar$, $z$方向自旋角动量$S_z=m_s\hbar$.

选择定则: $\Delta l=\pm1$, $\Delta m_l=0\,\text{或}\pm1$~($\,0\to0\,\text{禁戒}$).

    \chapter{望远镜\\Telescopes}

焦比(focal ratio)~$F$: 焦距与口径之比.

Airy斑\footnote{``斑的英文是disk.''}, $\theta\simeq1.22(\lambda/D)$.

主动光学(active optics): 消除由于热效应和望远镜移动时反射镜上重力变化造成的镜面变形. 每秒一次至百秒一次.

自适应光学(adaptive optics): 消除由于大气湍动造成的星像变形. 每秒数十次至每秒数百次.

补充内容:

Snell律: $n_{1\lambda}\sin\theta_1=n_{2\lambda}\sin\theta_2$, $n_\lambda:=c/v_\lambda$.

透镜靠近光轴处, lensmaker's formula:
\begin{equation*}
    \frac{1}{f_\lambda}=(n_\lambda-1)\left(\frac{1}{R_1}+\frac{1}{R_2}\right).
\end{equation*}
$f_\lambda$是凸透镜(converging lens)为正, 凹透镜(diverging lens)为负. $R$是镜面向外凸为正, 向内凹为负.

球面反射镜靠近光轴处$\left\lvert f\right\rvert=R/2$, 凹面镜(converging mirror)为正, 凸面镜(diverging mirror)为负.

近光轴, 底片比例尺(plate scale)~$\mathrm{d}\theta/\mathrm{d}y=1/f$.

衍射极限$\theta_\text{min}\simeq1.22(\lambda/D)$.

畸变(aberration):
\begin{itemize}
    \item 色差(chromatic aberration): 折射镜, 折射率不同, 焦距不一, 色散.
    \item 球差(spherical aberration): (球面镜)光不沿光轴时, 汇聚到光轴不同位置. 用抛物面镜解决.
    \item 彗差(coma): 抛物面焦距依赖于角度.
    \item 散光(astigmatism): 镜子不同部分焦点不同.
    \item 场曲(curvature of field): 设计镜子修正散光后, 焦面不是平面.
    \item 场变(distortion of field): 设计镜子修正散光后, 底片比例尺与光和光轴的距离相关, 图像变形.
\end{itemize}

照度(illumination)~$J$: 接收器单位面积单位时间接收到的能量.

$J\propto F^{-2}$.

焦点系统:
\begin{itemize}
    \item 主焦点(prime focus): 反射到中间.
    \item Newtonian: 加平面镜, 反射到旁边.
    \item Cassegrain: 主镜抛物面, 加凸面镜, 反射到主镜的洞里, 焦距增大.
    \item Ritchey--Chr\'etien: Cassegrain主镜抛物面$\to$双曲面.
    \item 折轴式(coud\'e): Cassegrain再加平面镜, 反射到旁边.
\end{itemize}

Schmidt: 球面+旋转四次曲面.

$1\,\text{Jy}=10^{-26}\,\text{W/(m}^2\!\cdot\text{Hz)}$. $1\,\text{erg}=10^{-7}\,\text{J}$.

    \chapter{双星和恒星参量\\Binary Systems and Stellar Parameters}

分类:
\begin{itemize}
    \item 光学双星(optical double): 假的.
    \item 视双星(visual binary): 都能看到, 可分辨开.
    \item 天体测量双星(astrometric binary): 一个可见, 振荡运动.
    \item 食双星(eclipsing binary): 有掩食.
    \item 光谱双星(spectrum binary): 两个堆叠的, 独立的, 可识别的光谱.
    \item 分光双星(spectroscopic binary): 谱线周期性运动.
\end{itemize}

质量函数(mass function): 分光双星, $Pv_{i\text{r},\text{max}}^3/2\pi G$.

补充内容: 

视双星. $i$: 长轴与天球切面夹角. $\tilde{\alpha}_1$, $\tilde{\alpha}_2$: 星和自己轨道对称中心的最大角距离. $d$: 双星-地球距离. $i$是可推定的.
\begin{equation*}
    \begin{cases}
        m_1 \tilde{\alpha}_1 = m_2 \tilde{\alpha}_2, \\
        a \cos i = (\tilde{\alpha}_1+\tilde{\alpha}_2) d. \\
    \end{cases}
\end{equation*}

分光双星, $e\simeq0$. $i$: 长轴与天球切面夹角. $v_{1\text{r},\text{max}}$, $v_{2\text{r},\text{max}}$: 星视向速度最大值. $P$: 双星运动周期. $\left\langle \sin^3 i\right\rangle \simeq 2/3$.
\begin{equation*}
    \begin{cases}
        m_1 v_{1\text{r},\text{max}} = m_2 v_{2\text{r},\text{max}}, \\
        2 \pi a = [(v_{1\text{r},\text{max}}+v_{2\text{r},\text{max}}) / \sin i] P. \\
    \end{cases}
\end{equation*}

分光食双星, $i\simeq90^{\circ}$, $e\simeq0$, $a\gg R$. $t_\text{a}$: 亮度开始下降到主极小的时刻. $t_\text{b}$: 亮度下降到主极小的时刻. $t_\text{c}$: 亮度开始从主极小上升的时刻. $v$: 双星相对速度.
\begin{equation*}
    \begin{cases}
        2 R_\text{小} = v (t_\text{b}-t_\text{a}), \\
        2 R_\text{大} = v (t_\text{c}-t_\text{a}). \\
    \end{cases}
\end{equation*}

食双星, $i\simeq90^{\circ}$, 忽略临边昏暗. $B_\text{max}$: 亮度极大值. $B_\text{pmin}$: 亮度主极小值. $B_\text{smin}$: 亮度次极小值.
\begin{equation*}
    \begin{cases}
        B_\text{max} \propto \pi R_\text{小}^2 \sigma T_\text{小}^4 + \pi R_\text{大}^2 \sigma T_\text{大}^4, \\
        B_\text{pmin} \propto \pi R_\text{大}^2 \sigma T_\text{大}^4 \\
        B_\text{smin} \propto \pi R_\text{小}^2 \sigma T_\text{小}^4 + (\pi R_\text{大}^2 - \pi R_\text{小}^2) \sigma T_\text{大}^4. \\
    \end{cases}
\end{equation*}

    \chapter{恒星光谱分类\\The Classification of Stellar Spectra}

Havard光谱分类: OBAFGKM, LT.

Morgan--Keenan(MK)光度级: Ia-O, Ia--Ib, II--V, VI(sd), D.

Boltzmann 方程:
\begin{equation*}
    P \propto g\,e^{-E/kT}.
\end{equation*}

Maxwell--Boltzmann速度分布函数: 根据Boltzmann 方程, 对于理想气体, $\mathrm{d}v_x\mathrm{d}v_y\mathrm{d}v_z$内的概率正比于$e^{-m(v_x^2+v_y^2+v_z^2)/2kT}$, 或者写成
\begin{equation*}
    p(v_x,v_y,v_z) \mathrm{d}v_x\mathrm{d}v_y\mathrm{d}v_z \propto e^{-\frac{m(v_x^2+v_y^2+v_z^2)}{2kT}} \mathrm{d}v_x\mathrm{d}v_y\mathrm{d}v_z.
\end{equation*}
Gauss积分, $e^{-\xi^2/2\sigma^2}$归一化系数是$1/\sqrt{2\pi}\sigma$, 于是乎有
\begin{equation*}
    p(v_x,v_y,v_z) \mathrm{d}v_x\mathrm{d}v_y\mathrm{d}v_z = \left(\frac{m}{2\pi kT}\right)^{3/2} e^{-\frac{m(v_x^2+v_y^2+v_z^2)}{2kT}} \mathrm{d}v_x\mathrm{d}v_y\mathrm{d}v_z.
\end{equation*}
这是速度分布, 转化到速率分布, 只需把对小方块$\mathrm{d}v_x\mathrm{d}v_y\mathrm{d}v_z$的积分变成对小球壳$4\pi v^2\mathrm{d}v$的积分, 于是乎有
\begin{equation*}
    p(v) \mathrm{d}v = \left(\frac{m}{2\pi kT}\right)^{3/2} e^{-\frac{mv^2}{2kT}} 4\pi v^2\mathrm{d}v.
\end{equation*}
变成数密度分布, 只需乘上数密度, 于是乎有
\begin{equation*}
    n_v \mathrm{d}v = n \left(\frac{m}{2\pi kT}\right)^{3/2} e^{-\frac{mv^2}{2kT}} 4\pi v^2\mathrm{d}v.
\end{equation*}
也就是说, $4\pi v^2\mathrm{d}v$是简并度$g$.

最可几速度(most probable speed): 求导.

方均根速度(root-mean-square speed): 能量均分定理$m\overline{v^2} /2=3(kT/2)$. 

Saha方程: 根据Boltzmann 方程,
\begin{equation*}
    \frac{N_{i+1}N_\text{e}}{N_{i}} = \frac{Z_{i+1}Z_\text{e}}{Z_{i}}.
\end{equation*}
其中
\begin{equation*}
    Z_{i+1} = \sum_j g_{i+1,j} e^{-E_{i+1,j}/kT},
\end{equation*}
\begin{equation*}
    Z_{i} = \sum_j g_{i,j} e^{-E_{i,j}/kT},
\end{equation*}
\begin{equation*}
    Z_\text{e} =  \int g_\text{e} e^{-[(p_x^2+p_y^2+p_z^2)/2m_\text{e}]/kT} \frac{\mathrm{d}x\mathrm{d}y\mathrm{d}z\mathrm{d}p_x\mathrm{d}p_y\mathrm{d}p_z}{h^3}.
\end{equation*}
上面三式都以束缚-自由临界状态能量为能量零点, 其中第三式运用了结论``一个微观状态等价于$xp_xyp_yzp_z$空间中体积为$h^3$的小区域''.
下面计算$Z_e$. $g_\text{e}=2$, 因此
\begin{equation*}
    Z_\text{e} = \frac{2\int\mathrm{d}x\mathrm{d}y\mathrm{d}z}{h^3} \int e^{-(p_x^2+p_y^2+p_z^2)/2m_\text{e}kT}\mathrm{d}p_x\mathrm{d}p_y\mathrm{d}p_z,
\end{equation*}
显然$V=\int\mathrm{d}x\mathrm{d}y\mathrm{d}z$, 后面是Gauss积分很好算, 结果就是
\begin{equation*}
    Z_\text{e} = \frac{2V}{h^3} (\sqrt{2\pi}\sqrt{m_\text{e}kT})^3 = 2 V \left(\frac{2\pi m_\text{e}kT}{h^2}\right)^{3/2} = 2 V \left(\frac{h}{\sqrt{2\pi m_\text{e}kT}}\right)^{-3}.
\end{equation*}
令
\begin{equation*}
    \widetilde{Z}_{i+1}  = \sum_j g_{i+1,j} e^{-(E_{i+1,j}-E_{i+1,1})/kT},
\end{equation*}
\begin{equation*}
    \widetilde{Z}_{i}  = \sum_j g_{i,j} e^{-(E_{i,j}-E_{i,1})/kT},
\end{equation*}
\begin{equation*}
    z_\text{e}  = 2 \left(\frac{2\pi m_\text{e}kT}{h^2}\right)^{3/2},
\end{equation*}
则由$\chi=E_{i+1,1}-E_{i,1}$和$n_\text{e}=N_\text{e}/V$可得
\begin{equation*}
    \frac{N_{i+1}n_\text{e}}{N_{i}} = \frac{\widetilde{Z}_{i+1}z_\text{e}}{\widetilde{Z}_{i}}e^{-\chi/kT}.
\end{equation*}
HII只有一种状态, $S=0$, 根据$S$和$Z$的关系可得$Z_\text{HII}=1$.

分光视差(spectroscopic parallax)~$d=100^{(m-M)/5}$, $m$当然可测, $M$通过看光谱比对H-R图可测, 和视差没有半毛关系!

    \chapter{恒星大气\\Stellar Atmospheres}

\begin{itemize}
    \item 有效温度(effective temperature): Stefan--Boltzmann律
    \item 激发温度(excitation temperature): Boltzmann方程.
    \item 电离温度(ionization temperature): Saha方程.
    \item 运动温度(kinetic temperature): Maxwell--Boltzmann分布.
    \item 色温度(color temperature): Planck律.
\end{itemize}

不透明度(opacity) $\kappa_\lambda$, $\mathrm{d}I_\lambda
=-I_\lambda\kappa_\lambda\rho\,\mathrm{d}s$, 单位$\text{m}^2/\text{kg}$.

光深(optical depth) $\tau_\lambda$, $\mathrm{d}\tau_\lambda=-\kappa_\lambda\rho\,\mathrm{d}s$, 无量纲.

辐射转移方程:
\begin{equation*}
    -\frac{1}{\kappa_\lambda\rho}\frac{\mathrm{d}I_\lambda}{\mathrm{d}s}
    =I_\lambda-S_\lambda.
\end{equation*}

等值宽度(equivalent width)~$W$,
\begin{equation*}
    W:=\int\frac{F_\text{c}-F(\lambda)}{F_\text{c}}\,\mathrm{d}\lambda.
\end{equation*}

\begin{itemize}
    \item 自然致宽: 不确定性原理, 激发态有寿命$\Delta t$ $\Rightarrow$激发态能量弥散$\Delta E$ $\Rightarrow$光子能量弥散$\Rightarrow$光子波长弥散, 似乎通常可以无视(Doppler致宽千分之一, 广义压强致宽同量级或低一量级).
    \item Doppler致宽: Maxwell--Boltzmann分布, 产生瘦高Doppler轮廓.
    \item 碰撞致宽: 原子和其他原子碰撞, 轨道改变而致宽.
    \item 压强致宽: 原子深入离子电场, 轨道改变而致宽, 广义包括碰撞致宽, 正比于数密度, 一同产生矮胖阻尼轮廓(又称Lorentz轮廓, 自然致宽也是这个轮廓).
\end{itemize}

补充内容:

比强度(specific density)~$I_\lambda$: 垂直于单位面积方向的单位立体角内单位时间通过的单位波长的能量. 对于黑体, $I_\lambda=B_\lambda$.

$\mathrm{d}\Omega=\mathrm{d}\phi(\sin\theta\,\mathrm{d}\theta)$.

平均强度(mean density)~$\left\langle I_\lambda\right\rangle $: 比强度对立体角求平均.
\begin{equation*}
    \left\langle I_\lambda\right\rangle = \frac{\int I_\lambda \,\mathrm{d}\Omega}{\int \,\mathrm{d}\Omega}=\frac{\int I_\lambda \,\mathrm{d}\Omega}{4\pi}.
\end{equation*}
对于黑体, $\left\langle I_\lambda\right\rangle=I_\lambda=B_\lambda$.

比能量密度(specific energy density)~$u_\lambda$: 首先假设只有一个方向的辐射强度, 取个垂直于这个方向的小面$\Delta A$, 在$\Delta t$时间内通过能量$I_\lambda \Delta A \Delta t$, 这些能量充斥了$\Delta A (c \Delta t)$的体积, 所以有$I_\lambda \Delta A \Delta t = u_\lambda \Delta A (c \Delta t)$, $u_\lambda = I_\lambda/c$. 可以证明任何小体元$V$内都有$\int_V u_\lambda = \int_V I_\lambda/c$. 现在辐射强度在任意方向都有, 所以要对立体角求和, 故有
\begin{equation*}
    u_\lambda = \int I_\lambda/c\,\mathrm{d}\Omega = 4\pi\left\langle I_\lambda\right\rangle/c.
\end{equation*}
对于黑体, $u=(4\sigma/c)\,T^4:=aT^4$.

比辐射流量(specific radiative flux)~$F_\lambda$: 把垂直于面元的辐射分量$I_\lambda\cos\theta$对立体角求和, 得到垂直于单位面积方向单位时间通过的单位波长的总能量, 即
\begin{equation*}
    F_\lambda = \int I_\lambda \cos\theta \,\mathrm{d}\Omega.
\end{equation*}
对于黑体, 只计算$\theta\le\pi/2$的部分, 可得$F=\sigma T^4=\pi \left\langle I_\lambda\right\rangle$.

辐射压强(radiation pressure)~$P_\lambda$: 反射情形下, $P_\lambda$要用辐射到板上的动量的法向分量的2倍来算. 首先, 由能量得到动量, 要除以$c$. 其次, $\theta$方向的辐射不垂直于板, $\Delta A$的实际有效面积只有$\Delta A \cos\theta$, 所以要乘以$\cos\theta$. 最后, 动量只取法向分量, 要再乘以$\cos\theta$. 只$\theta\le\pi/2$的部分有贡献, 所以有
\begin{equation*}
    P_\lambda = 2\int_{\theta\le\pi/2} I_\lambda\cos^2\theta/c \,\mathrm{d}\Omega.
\end{equation*}
对于透射情形, $P_\lambda$是面元$\theta<\pi/2$ 部分单位时间的动量改变量\footnote{单位时间的动量改变量不就是力么, 面元上的力不就是压强么\dots}. 首先$\theta<\pi/2$ 部分无论是吃光子还是吐光子, 动量改变都是一份, 所以不需要2倍的因子. 其次既要考虑进入$\theta<\pi/2$部分的光子(运动方向$\theta<\pi/2$)的动量也要考虑离开$\theta<\pi/2$部分的光子(运动方向$\theta>\pi/2$)的动量. 所以有
\begin{equation*}
    P_\lambda = \int I_\lambda\cos^2\theta/c \,\mathrm{d}\Omega.
\end{equation*}
对于黑体, $P=(1/3)\,u$.

对面源, 测得的是$I_\lambda$, 不随距离变化.

对点源, 测得的是$F_\lambda$, 和距离呈平方反比.

热动平衡(thermodynamics equilibrium): 所有正逆反应速率相同.

局部热动平衡(local thermodynamics equilibrium): 温度显著变化的距离大于粒子和光子的平均自由程, 粒子和光子不能逃出某范围, 在这范围内可以定义一个``这范围内的温度''.

平均自由程(mean free path)~$\ell$, 碰撞截面(collision cross section)~$\sigma$. $1/n=\sigma\ell$. $\sigma_\text{HI}=\pi(2a_\text{Bohr})^2$.

吸收系数(absorption coefficient)/不透明度\footnote{
    这里的不透明度是``质量不透明度'', 单位是$\text{m}^2/\text{kg}$, 乘上$\rho$是``体积不透明度'', 单位是$\text{m}^2/\text{m}^3$.
}(opacity)~$\kappa_\lambda$.

$\mathrm{d}I_\lambda
=-I_\lambda\kappa_\lambda\rho\,\mathrm{d}s$.
$\ell=1/\kappa_\lambda\rho$.

光深(optical depth)~$\tau_\lambda$.
$\mathrm{d}\tau_\lambda=-\kappa_\lambda\rho\,\mathrm{d}s$, $\mathrm{d}s=-\ell\,\mathrm{d}\tau_\lambda$, 无量纲.

\begin{itemize}
    \item Thomson散射: 自由电子.
    \item Compton散射: 高轨束缚电子, 光子动量远大于粒子动量, 光子能量远小于电子静能时$\to$~Thomson散射.
    \item Rayleigh散射: 高轨束缚电子, 光子动量远小于粒子动量, 粒子尺度远小于波长\footnote{粒子尺度与波长相当时是Mie散射.}, $\sigma \propto \lambda^{-4}$.
\end{itemize}

Kramers不透明度律: $\bar{\kappa}=\kappa_0\rho/T^{3.5}$.

$\kappa_0$近似常量, $\rho$单位$\text{kg}/\text{m}^3$, $T$单位$\text{K}$.

电子散射, $\bar{\kappa}=0.02\,(1+X)\,\text{m}^2/\text{kg}$.

随机行走(random walk), $d^2=N\ell^2$, $d=\ell\tau_\lambda$, $N=\tau_\lambda^2$, $\tau_\lambda\approx 2/3$原则.

发射系数(emission coefficient)~$j_\lambda$.

$\mathrm{d}I_\lambda
=j_\lambda\rho\,\mathrm{d}s$.

源函数(source function)~$S_\lambda=j_\lambda/\kappa_\lambda$. 对于黑体, $S_\lambda=B_\lambda$.

平面平行层大气(plane-parallel atmosphere), 垂直光深(vertical optical depth)~$\tau_{\lambda,\text{v}}$, 注意到$\kappa_\lambda$, $j_\lambda$, $S_\lambda$无方向性,
\begin{equation*}
    \frac{\mathrm{d}}{\mathrm{d}\tau_{\lambda,\text{v}}}I_\lambda\cos\theta=I_\lambda-S_\lambda.
\end{equation*}
灰大气(gray atmosphere), $\kappa_\lambda=\bar{\kappa}$. 对波长积分,
\begin{equation*}
    \frac{\mathrm{d}}{\mathrm{d}\tau_{\text{v}}}I\cos\theta=I-S.
\end{equation*}
上式对立体角积分和左右乘以$\cos\theta/c$后对立体角积分, 得
\begin{equation*}
    \begin{cases}
        \frac{\mathrm{d}}{\mathrm{d}\tau_{\text{v}}}F=4\pi \left\langle I\right\rangle -4\pi S , \\
        \frac{\mathrm{d}}{\mathrm{d}\tau_{\text{v}}}P=F/c-0 . \\
    \end{cases}
\end{equation*}
$F=\sigma T_\text{e}^4$不随$\tau_{\text{v}}$变化, 所以
\begin{equation*}
    \begin{cases}
        \left\langle I\right\rangle = S , \\
        P-P_0=(F/c)\tau_{\text{v}} . \\
    \end{cases}
\end{equation*}
Eddington近似: 向外$I_\text{out}$相同, 向内$I_\text{in}$相同, 则
\begin{equation*}
    \begin{cases}
        F=(\pi)(I_\text{out}-I_\text{in}) , \\
        P=(2\pi/3c)(I_\text{out}+I_\text{in}) . \\
    \end{cases}
\end{equation*}
$\tau_{\text{v}}$时$I_\text{in}=0$, 可得$P_0=(2\pi/3c)I_\text{out}=2F/3c$, 所以
\begin{equation*}
    P=(F/c)(\tau_{\text{v}}+2/3).
\end{equation*}
恰好有$P=(4\pi/3c)\left\langle I\right\rangle $, LTE, $\left\langle I\right\rangle=S=B=(\sigma/\pi)\,T^4$, 所以
\begin{equation*}
    T^4 = \frac{3}{4}(\tau_{\text{v}}+\frac{2}{3})T_\text{e}^4.
\end{equation*}

等值宽度(equivalent width)~$W$,
\begin{equation*}
    W:=\int\frac{F_\text{c}-F(\lambda)}{F_\text{c}}\,\mathrm{d}\lambda.
\end{equation*}
是令连续谱($F(\lambda)=F_\text{c}$)等于$1$后求等值宽度.

佛脱轮廓(Voigt profile): Doppler轮廓和阻尼(damping)轮廓的叠加. Doppler轮廓瘦高, 阻尼轮廓矮胖.

Schuster-Schwarzschild模型: 恒星光球是黑体辐射源; 光球外的原子产生吸收线.

柱密度(column density): 光球单位面积外面的原子数.

f值(f-value)/振子强度(oscillator strength): 从相同初态跃迁到不同终态的相对概率.

生长曲线: 自变量为能发生某跃迁的原子的柱密度(对数), 因变量为此跃迁产生的谱线的等值宽度(对数).

\end{document}
