% 编译方式: xelatex*2
\documentclass{ctexbook}
\usepackage{amsfonts}
\usepackage{amsmath}
\usepackage{amssymb}
\usepackage{hyperref}
\usepackage{syntonly}
%\syntaxonly
\pagestyle{plain}
\makeatletter
\newcommand{\starttoc}{
    \chapter*{\contentsname}
    \@starttoc{toc}
}
\makeatother
\renewcommand{\tableofcontents}{\twocolumn\starttoc\onecolumn}
\hypersetup{
	colorlinks,
	linkcolor=blue,
	filecolor=pink,
	urlcolor=cyan,
	citecolor=red,
}
\def\b{\boldsymbol}
\title{天文物理导论笔记}
\author{GasinAn}
\begin{document}
    \maketitle
    \noindent Copyright \textcopyright~2021 by GasinAn

\ 

\noindent All rights reserved. No part of this book may be reproduced, in any form or by any means, without permission in writing from the publisher, except by a BNUer.

\ 

\noindent The author and publisher of this book have used their best efforts in preparing this book. These efforts include the development, research, and testing of the theories, technologies and programs to determine their effectiveness. The author and publisher make no warranty of any kind, express or implied, with regard to these techniques or programs contained in this book. The author and publisher shall not be liable in any event of incidental or consequential damages in connection with, or arising out of, the furnishing, performance, or use of these techniques or programs.

\ 

\noindent Printed in China

    \tableofcontents
    \chapter{天球\\The Celestial Sphere}

会合周期(synodic period)~$S$.
\begin{equation*}
    1/S = \left\lvert 1/P - 1/P_\oplus \right\rvert.
\end{equation*}
通过定义自己推推就推出来了嘛$\!\text{\~{}}$

赤经(right ascension)~$\alpha$, 赤纬(declination)~$\delta$.
\begin{equation*}
    (\Delta\theta)^2 = (\Delta\alpha\cos\delta)^2+(\Delta\delta)^2
\end{equation*}
记忆方法: 当成勾股定理. 懒得写了, 自己想.

北天极的高度角等于地理纬度, 由此自己推算在天体上中天时刻, 天体赤纬, 天体天顶距和地理纬度的关系\footnote{这是本宝宝留给你们的作业题! 哼!}.

    \chapter{天体力学\\Celestial Mechanics}

看我PPT.

圆锥曲线(conic section)统一方程
\begin{equation*}
    r = \frac{ed}{1+e\cos\theta}.
\end{equation*}
椭圆(ellipse), $ed=a(1-e^2)$. 抛物线(parabola), $ed=2p$. 双曲线(hyperbola), $ed=a(1+e^2)$.

半长轴长(semimajor axis)~$a$, 半短轴长(semimajor axis)~$b$, 离心率(eccentricity)~$e$.

焦点(focal point), 近$\!\text{??}\!$点(perhelion), 远$\!\text{??}\!$点(aphelion).

椭圆面积$A=\pi ab$. 证明: 把单位圆横轴方向拉长$a$倍, 纵轴方向拉长$b$倍.

逃逸速度(escape velocity)~$v_\text{esc}$.
\begin{equation*}
    v_\text{esc} = \sqrt{2GM/r}.
\end{equation*}
第二宇宙速度$11.2\,\text{km/s}$.

质心(center of mass)~$\b{R}$.
\begin{equation*}
    \b{R} = \frac{
        \sum_{i=1}^n m_i\b{r}_i
    }{
        \sum_{i=1}^n m_i
    }.
\end{equation*}

折合质量(reduced mass)~$\mu$.
\begin{equation*}
    \mu = \frac{m_1m_2}{m_1+m_2}.
\end{equation*}
二体问题, 把坐标系建在其中一个天体上, 将其质量强行定为$M=m_1+m_2$, 另一天体质量强行定为$\mu$, 把日地系统的Kepler三定律, 和日地系统中地球的机械能和角动量的表达式中的$M_\odot$都换成$M$, $M_\oplus$都换成$\mu$, 就能得到二体问题的Kepler三定律和两天体的总机械能和角动量.

一些有用的表达式.
\begin{equation*}
    ed = \frac{1}{GM}\frac{L^2}{\mu^2}.
\end{equation*}
\begin{equation*}
    \mathrm{d}A=\frac{1}{2}\frac{L}{\mu}\mathrm{d}t
\end{equation*}
\begin{equation*}
    k=\frac{4\pi^2}{GM}.
\end{equation*}
完蛋了, 把Kepler三定律直接给出来了\dots~第一式推导: 计算perhelion处的$L$, $r$有了, $v$用机械能的表达式算. 第二式推导: 三角形面积是两条边的叉乘的长度的二分之一. 第三式推导: 假装轨道是圆的.

(老师没讲但很重要!)总机械能~$E$.
\begin{equation*}
    E =
    \begin{cases}
        -\frac{GM}{2a} & \text{椭圆}, \\
         0 & \text{抛物}, \\
         \frac{GM}{2a} & \text{双曲}. \\
    \end{cases}
\end{equation*}

维里定理(virial theorem): 系统, 平均总动能$\left\langle T\right\rangle$, 平均总势能$\left\langle V\right\rangle$, 平均总机械能$\left\langle E\right\rangle$,
\begin{equation*}
    2\left\langle T\right\rangle+\left\langle V\right\rangle = 0,
\end{equation*}
\begin{equation*}
    \left\langle E\right\rangle = \frac{1}{2}\left\langle V\right\rangle.
\end{equation*}
推论: 系统稳定, 平均总机械能必小于0.

    \chapter{光的连续谱\\The Continuous Spectrum of Light}

视差(parallax angle): 从天体上看, 地球和太阳的最大角距离.

$1\text{rad}\simeq 206265''$, $1\text{pc}\simeq 206265\text{AU}\simeq 3.26\text{ly}$.

视星等(apparent magnitude)~$m$. 辐射流量(radiant flux)~$F$: 仪器单位面积每秒接收到的能量. 光度(luminosity)~$L$: 天体每秒辐射的总能量.
\begin{equation*}
    m-m_0 = -2.5\lg \frac{F}{F_0}, F=100^{-(m-m_0)/5}F_0.
\end{equation*}
记忆法: 视星等五等, 亮度一百倍; 视星等越小越亮.
\begin{equation*}
    F = \frac{L}{4\pi r^2}.
\end{equation*}

绝对星等(absolute magnitude)~$M$: $10\text{pc}$处视星等. 距离模数(distance modulus)~$m-M$. 作业: 推导距离模数公式.

Stefan--Boltzmann律: $F=\sigma T^4$.~$B(T)=\sigma T^4/\pi$: 黑体垂直于单位面元方向单位立体角内单位时间辐射的能量.
\begin{equation*}
    F = \int_{\theta\in[0,\pi]}B\cos\theta\,\mathrm{d}\Omega.
\end{equation*}
Wien位移\footnote{~``位移''的英文是displacement.}律: $\lambda_\text{max}T\simeq(500\text{nm})(6000\text{K})$.

Planck函数: 
\begin{equation*}
    B_\nu(T)=\frac{2h\nu^3/c^2}{e^{h\nu/kT}-1}.
\end{equation*}
$B_\nu(T)$单位: $\text{J}/(\text{Hz}\cdot\text{s}\cdot\text{m}^2\cdot\text{sr})$. 作业: 推导$B_\lambda(T)$.

Planck函数的证明:
\begin{equation*}
    u_\nu(T)=\frac{4\pi}{c}B_\nu(T).
\end{equation*}
\begin{equation*}
    [u_\nu(T)\mathrm{d}\nu]V=\bar{\epsilon}(\nu)g(\nu)\mathrm{d}\nu.
\end{equation*}
\begin{equation*}
    \bar{\epsilon}(\nu)=\frac{h\nu}{e^{h\nu/kT}-1}=\sum_n (nh\nu)\frac{e^{-nh\nu/kT}}{\sum_n e^{-nh\nu/kT}}.
\end{equation*}
\begin{equation*}
    g(\nu)\mathrm{d}\nu=2\left[4\pi\left(\frac{L\nu}{c}\right)^2\mathrm{d}\left(\frac{L\nu}{c}\right)\right].
\end{equation*}

色指数(color index)~$X-Y$: 不同``波段''视星等的差, 等于不同``波段''绝对星等的差. 热星等(bolometric magnitude)~$m_\text{bol}$和$M_\text{bol}$: 全波段星等. 热改正(bolometric correction)~$BC=m_\text{bol}-V=M_\text{bol}-M_V$.

颜色-颜色图\footnote{不是色色图! 不是!}(color-color diagram): 横轴某色指数, 纵轴另一个. 黑体一条直线, 恒星线在黑体线下.

    \chapter{狭义相对论\\The Theory of Special Relativity}

太简单了, 没啥好记的.

    \chapter{光与物质的相互作用\\The Interaction of Light and Matter}

Kirchhoff律:
\begin{enumerate}
    \item 热致密气体或热固体产生连续谱, 无吸收线.
    \item 热弥漫气体带发射线.
    \item 冷弥漫气体在连续谱源前, 连续谱带吸收线.
\end{enumerate}

补充内容:

刻线间距$d$, 反射光与光栅法线夹角$\theta$, 光谱阶数$n$, 波长$\lambda$, $d\sin\theta=n\lambda$.

波长$\lambda$, 可分辨的最小波长差$\Delta\lambda$, 光谱阶数$n$, 刻线总数$N$, 分辨本领(resolving power)~$R=\lambda/\Delta\lambda=nN$.

Compton效应: 高能光子打低能(静止)电子, 光子波长变长. 逆Compton效应: 低能光子打高能电子, 光子波长变短. $\Delta\lambda=1-\cos\theta$.

$E_n=E_1/n^2$, $r_n=n^2r_1$. $E_1\simeq-13.6\text{eV}$, $r_1\simeq0.05\text{nm}$.

HI量子数$(n,l,m_l,m_s)$. 主量子数$n\in\mathbb{N}_+$, 轨道量子数$l=0,\dots,n-1$, 轨道磁量子数$m_l=-l,\dots,l$, 自旋磁量子数$m_s=-1/2,1/2$~($s=1/2$). 磁场方向为$z$方向, 轨道角动量$L=\sqrt{l(l+1)}\hbar$, $z$方向轨道角动量$L_z=m_l\hbar$, 自旋角动量$S=\sqrt{s(s+1)}\hbar=(\sqrt{3}/2)\hbar$, $z$方向自旋角动量$S_z=m_s\hbar$.

选择定则: $\Delta l=\pm1$, $\Delta m_l=0\,\text{或}\pm1$~($\,0\to0\,\text{禁戒}$).

    \chapter{望远镜\\Telescopes}

焦比(focal ratio)~$F$: 焦距与口径之比.

Airy斑\footnote{``斑的英文是disk.''}, $\theta\simeq1.22(\lambda/D)$.

主动光学(active optics): 消除由于热效应和望远镜移动时反射镜上重力变化造成的镜面变形. 每秒一次至百秒一次.

自适应光学(adaptive optics): 消除由于大气湍动造成的星像变形. 每秒数十次至每秒数百次.

补充内容:

Snell律: $n_{1\lambda}\sin\theta_1=n_{2\lambda}\sin\theta_2$, $n_\lambda:=c/v_\lambda$.

透镜靠近光轴处, lensmaker's formula:
\begin{equation*}
    \frac{1}{f_\lambda}=(n_\lambda-1)\left(\frac{1}{R_1}+\frac{1}{R_2}\right).
\end{equation*}
$f_\lambda$是凸透镜(converging lens)为正, 凹透镜(diverging lens)为负. $R$是镜面向外凸为正, 向内凹为负.

球面反射镜靠近光轴处$\left\lvert f\right\rvert=R/2$, 凹面镜(converging mirror)为正, 凸面镜(diverging mirror)为负.

近光轴, 底片比例尺(plate scale)~$\mathrm{d}\theta/\mathrm{d}y=1/f$.

衍射极限$\theta_\text{min}\simeq1.22(\lambda/D)$.

畸变(aberration):
\begin{itemize}
    \item 色差(chromatic aberration): 折射镜, 折射率不同, 焦距不一, 色散.
    \item 球差(spherical aberration): (球面镜)光不沿光轴时, 汇聚到光轴不同位置. 用抛物面镜解决.
    \item 彗差(coma): 抛物面焦距依赖于角度.
    \item 散光(astigmatism): 镜子不同部分焦点不同.
    \item 场曲(curvature of field): 设计镜子修正散光后, 焦面不是平面.
    \item 场变(distortion of field): 设计镜子修正散光后, 底片比例尺与光和光轴的距离相关, 图像变形.
\end{itemize}

照度(illumination)~$J$: 接收器单位面积单位时间接收到的能量.

$J\propto F^{-2}$.

焦点系统:
\begin{itemize}
    \item 主焦点(prime focus): 反射到中间.
    \item Newtonian: 加平面镜, 反射到旁边.
    \item Cassegrain: 主镜抛物面, 加凸面镜, 反射到主镜的洞里, 焦距增大.
    \item Ritchey--Chr\'etien: Cassegrain主镜抛物面$\to$双曲面.
    \item 折轴式(coud\'e): Cassegrain再加平面镜, 反射到旁边.
\end{itemize}

Schmidt: 球面+旋转四次曲面.

$1\,\text{Jy}=10^{-26}\,\text{W/(m}^2\!\cdot\text{Hz)}$. $1\,\text{erg}=10^{-7}\,\text{J}$.

\end{document}
