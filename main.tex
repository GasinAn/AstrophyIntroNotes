% 编译方式: xelatex*2
\documentclass{ctexbook}
\usepackage{amsfonts}
\usepackage{amsmath}
\usepackage{amssymb}
\usepackage{hyperref}
\usepackage{syntonly}
%\syntaxonly
\pagestyle{plain}
\makeatletter
\newcommand{\starttoc}{
    \chapter*{\contentsname}
    \@starttoc{toc}
}
\makeatother
%\renewcommand{\tableofcontents}{\twocolumn\starttoc\onecolumn}
\hypersetup{
	colorlinks,
	linkcolor=blue,
	filecolor=pink,
	urlcolor=cyan,
	citecolor=red,
}
\def\b{\boldsymbol}
\title{天文物理导论笔记}
\author{GasinAn}
\begin{document}
    \maketitle
    \noindent Copyright \textcopyright~2021 by GasinAn

\ 

\noindent All rights reserved. No part of this book may be reproduced, in any form or by any means, without permission in writing from the publisher, except by a BNUer.

\ 

\noindent The author and publisher of this book have used their best efforts in preparing this book. These efforts include the development, research, and testing of the theories, technologies and programs to determine their effectiveness. The author and publisher make no warranty of any kind, express or implied, with regard to these techniques or programs contained in this book. The author and publisher shall not be liable in any event of incidental or consequential damages in connection with, or arising out of, the furnishing, performance, or use of these techniques or programs.

\ 

\noindent Printed in China

    \tableofcontents
    \chapter{天球\\The Celestial Sphere}

会合周期(synodic period)~$S$.
\begin{equation*}
    1/S = \left\lvert 1/P - 1/P_\oplus \right\rvert.
\end{equation*}
通过定义自己推推就推出来了嘛$\!\text{\~{}}$

赤经(right ascension)~$\alpha$, 赤纬(declination)~$\delta$.
\begin{equation*}
    (\Delta\theta)^2 = (\Delta\alpha\cos\delta)^2+(\Delta\delta)^2
\end{equation*}
记忆方法: 当成勾股定理. 懒得写了, 自己想.

北天极的高度角等于地理纬度, 由此自己推算在天体上中天时刻, 天体赤纬, 天体天顶距和地理纬度的关系\footnote{这是本宝宝留给你们的作业题! 哼!}.

    \chapter{天体力学\\Celestial Mechanics}

看我PPT.

圆锥曲线(conic section)统一方程
\begin{equation*}
    r = \frac{ed}{1+e\cos\theta}.
\end{equation*}
椭圆(ellipse), $ed=a(1-e^2)$. 抛物线(parabola), $ed=2p$. 双曲线(hyperbola), $ed=a(1+e^2)$.

半长轴长(semimajor axis)~$a$, 半短轴长(semimajor axis)~$b$, 离心率(eccentricity)~$e$.

焦点(focal point), 近$\!\text{??}\!$点(perhelion), 远$\!\text{??}\!$点(aphelion).

椭圆面积$A=\pi ab$. 证明: 把单位圆横轴方向拉长$a$倍, 纵轴方向拉长$b$倍.

逃逸速度(escape velocity)~$v_\text{esc}$.
\begin{equation*}
    v_\text{esc} = \sqrt{2GM/r}.
\end{equation*}
第二宇宙速度$11.2\,\text{km/s}$.

质心(center of mass)~$\b{R}$.
\begin{equation*}
    \b{R} = \frac{
        \sum_{i=1}^n m_i\b{r}_i
    }{
        \sum_{i=1}^n m_i
    }.
\end{equation*}

折合质量(reduced mass)~$\mu$.
\begin{equation*}
    \mu = \frac{m_1m_2}{m_1+m_2}.
\end{equation*}
二体问题, 把坐标系建在其中一个天体上, 将其质量强行定为$M=m_1+m_2$, 另一天体质量强行定为$\mu$, 把日地系统的Kepler三定律, 和日地系统中地球的机械能和角动量的表达式中的$M_\odot$都换成$M$, $M_\oplus$都换成$\mu$, 就能得到二体问题的Kepler三定律和两天体的总机械能和角动量.

一些有用的表达式.
\begin{equation*}
    ed = \frac{1}{GM}\frac{L^2}{\mu^2}.
\end{equation*}
\begin{equation*}
    \mathrm{d}A=\frac{1}{2}\frac{L}{\mu}\mathrm{d}t
\end{equation*}
\begin{equation*}
    k=\frac{4\pi^2}{GM}.
\end{equation*}
完蛋了, 把Kepler三定律直接给出来了\dots~第一式推导: 计算perhelion处的$L$, $r$有了, $v$用机械能的表达式算. 第二式推导: 三角形面积是两条边的叉乘的长度的二分之一. 第三式推导: 假装轨道是圆的.

(老师没讲但很重要!)总机械能~$E$.
\begin{equation*}
    E =
    \begin{cases}
        -\frac{GM}{2a} & \text{椭圆}, \\
         0 & \text{抛物}, \\
         \frac{GM}{2a} & \text{双曲}. \\
    \end{cases}
\end{equation*}

维里定理(virial theorem): 系统, 平均总动能$\left\langle T\right\rangle$, 平均总势能$\left\langle V\right\rangle$, 平均总机械能$\left\langle E\right\rangle$,
\begin{equation*}
    2\left\langle T\right\rangle+\left\langle V\right\rangle = 0,
\end{equation*}
\begin{equation*}
    \left\langle E\right\rangle = \frac{1}{2}\left\langle V\right\rangle.
\end{equation*}
推论: 系统稳定, 平均总机械能必小于0.

    \chapter{光的连续谱\\The Continuous Spectrum of Light}

视差(parallax angle): 从天体上看, 地球和太阳的最大角距离.

$1\text{rad}\simeq 206265''$, $1\text{pc}\simeq 206265\text{AU}\simeq 3.26\text{ly}$.

视星等(apparent magnitude)~$m$. 辐射流量(radiant flux)~$F$: 仪器单位面积每秒接收到的能量. 光度(luminosity)~$L$: 天体每秒辐射的总能量.
\begin{equation*}
    m-m_0 = -2.5\lg \frac{F}{F_0}, F=100^{-(m-m_0)/5}F_0.
\end{equation*}
记忆法: 视星等五等, 亮度一百倍; 视星等越小越亮.
\begin{equation*}
    F = \frac{L}{4\pi r^2}.
\end{equation*}

绝对星等(absolute magnitude)~$M$: $10\text{pc}$处视星等. 距离模数(distance modulus)~$m-M$. 作业: 推导距离模数公式.

Stefan--Boltzmann律: $F=\sigma T^4$.~$B(T)=\sigma T^4/\pi$: 黑体垂直于单位面元方向单位立体角内单位时间辐射的能量.
\begin{equation*}
    F = \int_{\theta\in[0,\pi]}B\cos\theta\,\mathrm{d}\Omega.
\end{equation*}
Wien位移\footnote{~``位移''的英文是displacement.}律: $\lambda_\text{max}T\simeq(500\text{nm})(6000\text{K})$.

Planck函数: 
\begin{equation*}
    B_\nu(T)=\frac{2h\nu^3/c^2}{e^{h\nu/kT}-1}.
\end{equation*}
$B_\nu(T)$单位: $\text{J}/(\text{Hz}\cdot\text{s}\cdot\text{m}^2\cdot\text{sr})$. 作业: 推导$B_\lambda(T)$.

Planck函数的证明:
\begin{equation*}
    u_\nu(T)=\frac{4\pi}{c}B_\nu(T).
\end{equation*}
\begin{equation*}
    [u_\nu(T)\mathrm{d}\nu]V=\bar{\epsilon}(\nu)g(\nu)\mathrm{d}\nu.
\end{equation*}
\begin{equation*}
    \bar{\epsilon}(\nu)=\frac{h\nu}{e^{h\nu/kT}-1}=\sum_n (nh\nu)\frac{e^{-nh\nu/kT}}{\sum_n e^{-nh\nu/kT}}.
\end{equation*}
\begin{equation*}
    g(\nu)\mathrm{d}\nu=2\left[4\pi\left(\frac{L\nu}{c}\right)^2\mathrm{d}\left(\frac{L\nu}{c}\right)\right].
\end{equation*}

色指数(color index)~$X-Y$: 不同``波段''视星等的差, 等于不同``波段''绝对星等的差. 热星等(bolometric magnitude)~$m_\text{bol}$和$M_\text{bol}$: 全波段星等. 热改正(bolometric correction)~$BC=m_\text{bol}-V=M_\text{bol}-M_V$.

颜色-颜色图\footnote{不是色色图! 不是!}(color-color diagram): 横轴某色指数, 纵轴另一个. 黑体一条直线, 恒星线在黑体线下.

    \chapter{狭义相对论\\The Theory of Special Relativity}

太简单了, 没啥好记的.

    \chapter{光与物质的相互作用\\The Interaction of Light and Matter}

Kirchhoff律:
\begin{enumerate}
    \item 热致密气体或热固体产生连续谱, 无吸收线.
    \item 热弥漫气体带发射线.
    \item 冷弥漫气体在连续谱源前, 连续谱带吸收线.
\end{enumerate}

补充内容:

刻线间距$d$, 反射光与光栅法线夹角$\theta$, 光谱阶数$n$, 波长$\lambda$, $d\sin\theta=n\lambda$.

波长$\lambda$, 可分辨的最小波长差$\Delta\lambda$, 光谱阶数$n$, 刻线总数$N$, 分辨本领(resolving power)~$R=\lambda/\Delta\lambda=nN$.

Compton效应: 高能光子打低能(静止)电子, 光子波长变长. 逆Compton效应: 低能光子打高能电子, 光子波长变短. $\Delta\lambda=1-\cos\theta$.

$E_n=E_1/n^2$, $r_n=n^2r_1$. $E_1\simeq-13.6\text{eV}$, $r_1\simeq0.05\text{nm}$.

HI量子数$(n,l,m_l,m_s)$. 主量子数$n\in\mathbb{N}_+$, 轨道量子数$l=0,\dots,n-1$, 轨道磁量子数$m_l=-l,\dots,l$, 自旋磁量子数$m_s=-1/2,1/2$~($s=1/2$). 磁场方向为$z$方向, 轨道角动量$L=\sqrt{l(l+1)}\hbar$, $z$方向轨道角动量$L_z=m_l\hbar$, 自旋角动量$S=\sqrt{s(s+1)}\hbar=(\sqrt{3}/2)\hbar$, $z$方向自旋角动量$S_z=m_s\hbar$.

选择定则: $\Delta l=\pm1$, $\Delta m_l=0\,\text{或}\pm1$~($\,0\to0\,\text{禁戒}$).

    \chapter{望远镜\\Telescopes}

焦比(focal ratio)~$F$: 焦距与口径之比.

Airy斑\footnote{``斑的英文是disk.''}, $\theta\simeq1.22(\lambda/D)$.

主动光学(active optics): 消除由于热效应和望远镜移动时反射镜上重力变化造成的镜面变形. 每秒一次至百秒一次.

自适应光学(adaptive optics): 消除由于大气湍动造成的星像变形. 每秒数十次至每秒数百次.

补充内容:

Snell律: $n_{1\lambda}\sin\theta_1=n_{2\lambda}\sin\theta_2$, $n_\lambda:=c/v_\lambda$.

透镜靠近光轴处, lensmaker's formula:
\begin{equation*}
    \frac{1}{f_\lambda}=(n_\lambda-1)\left(\frac{1}{R_1}+\frac{1}{R_2}\right).
\end{equation*}
$f_\lambda$是凸透镜(converging lens)为正, 凹透镜(diverging lens)为负. $R$是镜面向外凸为正, 向内凹为负.

球面反射镜靠近光轴处$\left\lvert f\right\rvert=R/2$, 凹面镜(converging mirror)为正, 凸面镜(diverging mirror)为负.

近光轴, 底片比例尺(plate scale)~$\mathrm{d}\theta/\mathrm{d}y=1/f$.

衍射极限$\theta_\text{min}\simeq1.22(\lambda/D)$.

畸变(aberration):
\begin{itemize}
    \item 色差(chromatic aberration): 折射镜, 折射率不同, 焦距不一, 色散.
    \item 球差(spherical aberration): (球面镜)光不沿光轴时, 汇聚到光轴不同位置. 用抛物面镜解决.
    \item 彗差(coma): 抛物面焦距依赖于角度.
    \item 散光(astigmatism): 镜子不同部分焦点不同.
    \item 场曲(curvature of field): 设计镜子修正散光后, 焦面不是平面.
    \item 场变(distortion of field): 设计镜子修正散光后, 底片比例尺与光和光轴的距离相关, 图像变形.
\end{itemize}

照度(illumination)~$J$: 接收器单位面积单位时间接收到的能量.

$J\propto F^{-2}$.

焦点系统:
\begin{itemize}
    \item 主焦点(prime focus): 反射到中间.
    \item Newtonian: 加平面镜, 反射到旁边.
    \item Cassegrain: 主镜抛物面, 加凸面镜, 反射到主镜的洞里, 焦距增大.
    \item Ritchey--Chr\'etien: Cassegrain主镜抛物面$\to$双曲面.
    \item 折轴式(coud\'e): Cassegrain再加平面镜, 反射到旁边.
\end{itemize}

Schmidt: 球面+旋转四次曲面.

$1\,\text{Jy}=10^{-26}\,\text{W/(m}^2\!\cdot\text{Hz)}$. $1\,\text{erg}=10^{-7}\,\text{J}$.

    \chapter{双星和恒星参量\\Binary Systems and Stellar Parameters}

分类:
\begin{itemize}
    \item 光学双星(optical double): 假的.
    \item 视双星(visual binary): 都能看到, 可分辨开.
    \item 天体测量双星(astrometric binary): 一个可见, 振荡运动.
    \item 食双星(eclipsing binary): 有掩食.
    \item 光谱双星(spectrum binary): 两个堆叠的, 独立的, 可识别的光谱.
    \item 分光双星(spectroscopic binary): 谱线周期性运动.
\end{itemize}

质量函数(mass function): 分光双星, $Pv_{i\text{r},\text{max}}^3/2\pi G$.

补充内容: 

视双星. $i$: 长轴与天球切面夹角. $\tilde{\alpha}_1$, $\tilde{\alpha}_2$: 星和自己轨道对称中心的最大角距离. $d$: 双星-地球距离. $i$是可推定的.
\begin{equation*}
    \begin{cases}
        m_1 \tilde{\alpha}_1 = m_2 \tilde{\alpha}_2, \\
        a \cos i = (\tilde{\alpha}_1+\tilde{\alpha}_2) d. \\
    \end{cases}
\end{equation*}

分光双星, $e\simeq0$. $i$: 长轴与天球切面夹角. $v_{1\text{r},\text{max}}$, $v_{2\text{r},\text{max}}$: 星视向速度最大值. $P$: 双星运动周期. $\left\langle \sin^3 i\right\rangle \simeq 2/3$.
\begin{equation*}
    \begin{cases}
        m_1 v_{1\text{r},\text{max}} = m_2 v_{2\text{r},\text{max}}, \\
        2 \pi a = [(v_{1\text{r},\text{max}}+v_{2\text{r},\text{max}}) / \sin i] P. \\
    \end{cases}
\end{equation*}

分光食双星, $i\simeq90^{\circ}$, $e\simeq0$, $a\gg R$. $t_\text{a}$: 亮度开始下降到主极小的时刻. $t_\text{b}$: 亮度下降到主极小的时刻. $t_\text{c}$: 亮度开始从主极小上升的时刻. $v$: 双星相对速度.
\begin{equation*}
    \begin{cases}
        2 R_\text{小} = v (t_\text{b}-t_\text{a}), \\
        2 R_\text{大} = v (t_\text{c}-t_\text{a}). \\
    \end{cases}
\end{equation*}

食双星, $i\simeq90^{\circ}$, 忽略临边昏暗. $B_\text{max}$: 亮度极大值. $B_\text{pmin}$: 亮度主极小值. $B_\text{smin}$: 亮度次极小值.
\begin{equation*}
    \begin{cases}
        B_\text{max} \propto \pi R_\text{小}^2 \sigma T_\text{小}^4 + \pi R_\text{大}^2 \sigma T_\text{大}^4, \\
        B_\text{pmin} \propto \pi R_\text{大}^2 \sigma T_\text{大}^4 \\
        B_\text{smin} \propto \pi R_\text{小}^2 \sigma T_\text{小}^4 + (\pi R_\text{大}^2 - \pi R_\text{小}^2) \sigma T_\text{大}^4. \\
    \end{cases}
\end{equation*}

    \chapter{恒星光谱分类\\The Classification of Stellar Spectra}

Havard光谱分类: OBAFGKM, LT.

Morgan--Keenan(MK)光度级: Ia-O, Ia--Ib, II--V, VI(sd), D.

Boltzmann 方程:
\begin{equation*}
    P \propto g\,e^{-E/kT}.
\end{equation*}

Maxwell--Boltzmann速度分布函数: 根据Boltzmann 方程, 对于理想气体, $\mathrm{d}v_x\mathrm{d}v_y\mathrm{d}v_z$内的概率正比于$e^{-m(v_x^2+v_y^2+v_z^2)/2kT}$, 或者写成
\begin{equation*}
    p(v_x,v_y,v_z) \mathrm{d}v_x\mathrm{d}v_y\mathrm{d}v_z \propto e^{-\frac{m(v_x^2+v_y^2+v_z^2)}{2kT}} \mathrm{d}v_x\mathrm{d}v_y\mathrm{d}v_z.
\end{equation*}
Gauss积分, $e^{-\xi^2/2\sigma^2}$归一化系数是$1/\sqrt{2\pi}\sigma$, 于是乎有
\begin{equation*}
    p(v_x,v_y,v_z) \mathrm{d}v_x\mathrm{d}v_y\mathrm{d}v_z = \left(\frac{m}{2\pi kT}\right)^{3/2} e^{-\frac{m(v_x^2+v_y^2+v_z^2)}{2kT}} \mathrm{d}v_x\mathrm{d}v_y\mathrm{d}v_z.
\end{equation*}
这是速度分布, 转化到速率分布, 只需把对小方块$\mathrm{d}v_x\mathrm{d}v_y\mathrm{d}v_z$的积分变成对小球壳$4\pi v^2\mathrm{d}v$的积分, 于是乎有
\begin{equation*}
    p(v) \mathrm{d}v = \left(\frac{m}{2\pi kT}\right)^{3/2} e^{-\frac{mv^2}{2kT}} 4\pi v^2\mathrm{d}v.
\end{equation*}
变成数密度分布, 只需乘上数密度, 于是乎有
\begin{equation*}
    n_v \mathrm{d}v = n \left(\frac{m}{2\pi kT}\right)^{3/2} e^{-\frac{mv^2}{2kT}} 4\pi v^2\mathrm{d}v.
\end{equation*}
也就是说, $4\pi v^2\mathrm{d}v$是简并度$g$.

最可几速度(most probable speed): 求导.

方均根速度(root-mean-square speed): 能量均分定理$m\overline{v^2} /2=3(kT/2)$. 

Saha方程: 根据Boltzmann 方程,
\begin{equation*}
    \frac{N_{i+1}N_\text{e}}{N_{i}} = \frac{Z_{i+1}Z_\text{e}}{Z_{i}}.
\end{equation*}
其中
\begin{equation*}
    Z_{i+1} = \sum_j g_{i+1,j} e^{-E_{i+1,j}/kT},
\end{equation*}
\begin{equation*}
    Z_{i} = \sum_j g_{i,j} e^{-E_{i,j}/kT},
\end{equation*}
\begin{equation*}
    Z_\text{e} =  \int g_\text{e} e^{-[(p_x^2+p_y^2+p_z^2)/2m_\text{e}]/kT} \frac{\mathrm{d}x\mathrm{d}y\mathrm{d}z\mathrm{d}p_x\mathrm{d}p_y\mathrm{d}p_z}{h^3}.
\end{equation*}
上面三式都以束缚-自由临界状态能量为能量零点, 其中第三式运用了结论``一个微观状态等价于$xp_xyp_yzp_z$空间中体积为$h^3$的小区域''.
下面计算$Z_e$. $g_\text{e}=2$, 因此
\begin{equation*}
    Z_\text{e} = \frac{2\int\mathrm{d}x\mathrm{d}y\mathrm{d}z}{h^3} \int e^{-(p_x^2+p_y^2+p_z^2)/2m_\text{e}kT}\mathrm{d}p_x\mathrm{d}p_y\mathrm{d}p_z,
\end{equation*}
显然$V=\int\mathrm{d}x\mathrm{d}y\mathrm{d}z$, 后面是Gauss积分很好算, 结果就是
\begin{equation*}
    Z_\text{e} = \frac{2V}{h^3} (\sqrt{2\pi}\sqrt{m_\text{e}kT})^3 = 2 V \left(\frac{2\pi m_\text{e}kT}{h^2}\right)^{3/2} = 2 V \left(\frac{h}{\sqrt{2\pi m_\text{e}kT}}\right)^{-3}.
\end{equation*}
令
\begin{equation*}
    \widetilde{Z}_{i+1}  = \sum_j g_{i+1,j} e^{-(E_{i+1,j}-E_{i+1,1})/kT},
\end{equation*}
\begin{equation*}
    \widetilde{Z}_{i}  = \sum_j g_{i,j} e^{-(E_{i,j}-E_{i,1})/kT},
\end{equation*}
\begin{equation*}
    z_\text{e}  = 2 \left(\frac{2\pi m_\text{e}kT}{h^2}\right)^{3/2},
\end{equation*}
则由$\chi=E_{i+1,1}-E_{i,1}$和$n_\text{e}=N_\text{e}/V$可得
\begin{equation*}
    \frac{N_{i+1}n_\text{e}}{N_{i}} = \frac{\widetilde{Z}_{i+1}z_\text{e}}{\widetilde{Z}_{i}}e^{-\chi/kT}.
\end{equation*}
HII只有一种状态, $S=0$, 根据$S$和$Z$的关系可得$Z_\text{HII}=1$.

分光视差(spectroscopic parallax)~$d=100^{(m-M)/5}$, $m$当然可测, $M$通过看光谱比对H-R图可测, 和视差没有半毛关系!

    \chapter{恒星大气\\Stellar Atmospheres}

\begin{itemize}
    \item 有效温度(effective temperature): Stefan--Boltzmann律
    \item 激发温度(excitation temperature): Boltzmann方程.
    \item 电离温度(ionization temperature): Saha方程.
    \item 运动温度(kinetic temperature): Maxwell--Boltzmann分布.
    \item 色温度(color temperature): Planck律.
\end{itemize}

不透明度(opacity) $\kappa_\lambda$, $\mathrm{d}I_\lambda
=-I_\lambda\kappa_\lambda\rho\,\mathrm{d}s$, 单位$\text{m}^2/\text{kg}$.

光深(optical depth) $\tau_\lambda$, $\mathrm{d}\tau_\lambda=-\kappa_\lambda\rho\,\mathrm{d}s$, 无量纲.

辐射转移方程:
\begin{equation*}
    -\frac{1}{\kappa_\lambda\rho}\frac{\mathrm{d}I_\lambda}{\mathrm{d}s}
    =I_\lambda-S_\lambda.
\end{equation*}

等值宽度(equivalent width)~$W$,
\begin{equation*}
    W:=\int\frac{F_\text{c}-F(\lambda)}{F_\text{c}}\,\mathrm{d}\lambda.
\end{equation*}

\begin{itemize}
    \item 自然致宽: 不确定性原理, 激发态有寿命$\Delta t$ $\Rightarrow$激发态能量弥散$\Delta E$ $\Rightarrow$光子能量弥散$\Rightarrow$光子波长弥散, 似乎通常可以无视(Doppler致宽千分之一, 广义压强致宽同量级或低一量级).
    \item Doppler致宽: Maxwell--Boltzmann分布, 产生瘦高Doppler轮廓.
    \item 碰撞致宽: 原子和其他原子碰撞, 轨道改变而致宽.
    \item 压强致宽: 原子深入离子电场, 轨道改变而致宽, 广义包括碰撞致宽, 正比于数密度, 一同产生矮胖阻尼轮廓(又称Lorentz轮廓, 自然致宽也是这个轮廓).
\end{itemize}

补充内容:

比强度(specific density)~$I_\lambda$: 垂直于单位面积方向的单位立体角内单位时间通过的单位波长的能量. 对于黑体, $I_\lambda=B_\lambda$.

$\mathrm{d}\Omega=\mathrm{d}\phi(\sin\theta\,\mathrm{d}\theta)$.

平均强度(mean density)~$\left\langle I_\lambda\right\rangle $: 比强度对立体角求平均.
\begin{equation*}
    \left\langle I_\lambda\right\rangle = \frac{\int I_\lambda \,\mathrm{d}\Omega}{\int \,\mathrm{d}\Omega}=\frac{\int I_\lambda \,\mathrm{d}\Omega}{4\pi}.
\end{equation*}
对于黑体, $\left\langle I_\lambda\right\rangle=I_\lambda=B_\lambda$.

比能量密度(specific energy density)~$u_\lambda$: 首先假设只有一个方向的辐射强度, 取个垂直于这个方向的小面$\Delta A$, 在$\Delta t$时间内通过能量$I_\lambda \Delta A \Delta t$, 这些能量充斥了$\Delta A (c \Delta t)$的体积, 所以有$I_\lambda \Delta A \Delta t = u_\lambda \Delta A (c \Delta t)$, $u_\lambda = I_\lambda/c$. 可以证明任何小体元$V$内都有$\int_V u_\lambda = \int_V I_\lambda/c$. 现在辐射强度在任意方向都有, 所以要对立体角求和, 故有
\begin{equation*}
    u_\lambda = \int I_\lambda/c\,\mathrm{d}\Omega = 4\pi\left\langle I_\lambda\right\rangle/c.
\end{equation*}
对于黑体, $u=(4\sigma/c)\,T^4:=aT^4$.

比辐射流量(specific radiative flux)~$F_\lambda$: 把垂直于面元的辐射分量$I_\lambda\cos\theta$对立体角求和, 得到垂直于单位面积方向单位时间通过的单位波长的总能量, 即
\begin{equation*}
    F_\lambda = \int I_\lambda \cos\theta \,\mathrm{d}\Omega.
\end{equation*}
对于黑体, 只计算$\theta\le\pi/2$的部分, 可得$F=\sigma T^4=\pi \left\langle I_\lambda\right\rangle$.

辐射压强(radiation pressure)~$P_\lambda$: 反射情形下, $P_\lambda$要用辐射到板上的动量的法向分量的2倍来算. 首先, 由能量得到动量, 要除以$c$. 其次, $\theta$方向的辐射不垂直于板, $\Delta A$的实际有效面积只有$\Delta A \cos\theta$, 所以要乘以$\cos\theta$. 最后, 动量只取法向分量, 要再乘以$\cos\theta$. 只$\theta\le\pi/2$的部分有贡献, 所以有
\begin{equation*}
    P_\lambda = 2\int_{\theta\le\pi/2} I_\lambda\cos^2\theta/c \,\mathrm{d}\Omega.
\end{equation*}
对于透射情形, $P_\lambda$是面元$\theta<\pi/2$ 部分单位时间的动量改变量\footnote{单位时间的动量改变量不就是力么, 面元上的力不就是压强么\dots}. 首先$\theta<\pi/2$ 部分无论是吃光子还是吐光子, 动量改变都是一份, 所以不需要2倍的因子. 其次既要考虑进入$\theta<\pi/2$部分的光子(运动方向$\theta<\pi/2$)的动量也要考虑离开$\theta<\pi/2$部分的光子(运动方向$\theta>\pi/2$)的动量. 所以有
\begin{equation*}
    P_\lambda = \int I_\lambda\cos^2\theta/c \,\mathrm{d}\Omega.
\end{equation*}
对于黑体, $P=(1/3)\,u$.

对面源, 测得的是$I_\lambda$, 不随距离变化.

对点源, 测得的是$F_\lambda$, 和距离呈平方反比.

热动平衡(thermodynamics equilibrium): 所有正逆反应速率相同.

局部热动平衡(local thermodynamics equilibrium): 温度显著变化的距离大于粒子和光子的平均自由程, 粒子和光子不能逃出某范围, 在这范围内可以定义一个``这范围内的温度''.

平均自由程(mean free path)~$\ell$, 碰撞截面(collision cross section)~$\sigma$. $1/n=\sigma\ell$. $\sigma_\text{HI}=\pi(2a_\text{Bohr})^2$.

吸收系数(absorption coefficient)/不透明度\footnote{
    这里的不透明度是``质量不透明度'', 单位是$\text{m}^2/\text{kg}$, 乘上$\rho$是``体积不透明度'', 单位是$\text{m}^2/\text{m}^3$.
}(opacity)~$\kappa_\lambda$.

$\mathrm{d}I_\lambda
=-I_\lambda\kappa_\lambda\rho\,\mathrm{d}s$.
$\ell=1/\kappa_\lambda\rho$.

光深(optical depth)~$\tau_\lambda$.
$\mathrm{d}\tau_\lambda=-\kappa_\lambda\rho\,\mathrm{d}s$, $\mathrm{d}s=-\ell\,\mathrm{d}\tau_\lambda$, 无量纲.

\begin{itemize}
    \item Thomson散射: 自由电子.
    \item Compton散射: 高轨束缚电子, 光子动量远大于粒子动量, 光子能量远小于电子静能时$\to$~Thomson散射.
    \item Rayleigh散射: 高轨束缚电子, 光子动量远小于粒子动量, 粒子尺度远小于波长\footnote{粒子尺度与波长相当时是Mie散射.}, $\sigma \propto \lambda^{-4}$.
\end{itemize}

Kramers不透明度律: $\bar{\kappa}=\kappa_0\rho/T^{3.5}$.

$\kappa_0$近似常量, $\rho$单位$\text{kg}/\text{m}^3$, $T$单位$\text{K}$.

电子散射, $\bar{\kappa}=0.02\,(1+X)\,\text{m}^2/\text{kg}$.

随机行走(random walk), $d^2=N\ell^2$, $d=\ell\tau_\lambda$, $N=\tau_\lambda^2$, $\tau_\lambda\approx 2/3$原则.

发射系数(emission coefficient)~$j_\lambda$.

$\mathrm{d}I_\lambda
=j_\lambda\rho\,\mathrm{d}s$.

源函数(source function)~$S_\lambda=j_\lambda/\kappa_\lambda$. 对于黑体, $S_\lambda=B_\lambda$.

平面平行层大气(plane-parallel atmosphere), 垂直光深(vertical optical depth)~$\tau_{\lambda,\text{v}}$, 注意到$\kappa_\lambda$, $j_\lambda$, $S_\lambda$无方向性,
\begin{equation*}
    \frac{\mathrm{d}}{\mathrm{d}\tau_{\lambda,\text{v}}}I_\lambda\cos\theta=I_\lambda-S_\lambda.
\end{equation*}
灰大气(gray atmosphere), $\kappa_\lambda=\bar{\kappa}$. 对波长积分,
\begin{equation*}
    \frac{\mathrm{d}}{\mathrm{d}\tau_{\text{v}}}I\cos\theta=I-S.
\end{equation*}
上式对立体角积分和左右乘以$\cos\theta/c$后对立体角积分, 得
\begin{equation*}
    \begin{cases}
        \frac{\mathrm{d}}{\mathrm{d}\tau_{\text{v}}}F=4\pi \left\langle I\right\rangle -4\pi S , \\
        \frac{\mathrm{d}}{\mathrm{d}\tau_{\text{v}}}P=F/c-0 . \\
    \end{cases}
\end{equation*}
$F=\sigma T_\text{e}^4$不随$\tau_{\text{v}}$变化, 所以
\begin{equation*}
    \begin{cases}
        \left\langle I\right\rangle = S , \\
        P-P_0=(F/c)\tau_{\text{v}} . \\
    \end{cases}
\end{equation*}
Eddington近似: 向外$I_\text{out}$相同, 向内$I_\text{in}$相同, 则
\begin{equation*}
    \begin{cases}
        F=(\pi)(I_\text{out}-I_\text{in}) , \\
        P=(2\pi/3c)(I_\text{out}+I_\text{in}) . \\
    \end{cases}
\end{equation*}
$\tau_{\text{v}}$时$I_\text{in}=0$, 可得$P_0=(2\pi/3c)I_\text{out}=2F/3c$, 所以
\begin{equation*}
    P=(F/c)(\tau_{\text{v}}+2/3).
\end{equation*}
恰好有$P=(4\pi/3c)\left\langle I\right\rangle $, LTE, $\left\langle I\right\rangle=S=B=(\sigma/\pi)\,T^4$, 所以
\begin{equation*}
    T^4 = \frac{3}{4}(\tau_{\text{v}}+\frac{2}{3})T_\text{e}^4.
\end{equation*}

等值宽度(equivalent width)~$W$,
\begin{equation*}
    W:=\int\frac{F_\text{c}-F(\lambda)}{F_\text{c}}\,\mathrm{d}\lambda.
\end{equation*}
是令连续谱($F(\lambda)=F_\text{c}$)等于$1$后求等值宽度.

佛脱轮廓(Voigt profile): Doppler轮廓和阻尼(damping)轮廓的叠加. Doppler轮廓瘦高, 阻尼轮廓矮胖.

Schuster-Schwarzschild模型: 恒星光球是黑体辐射源; 光球外的原子产生吸收线.

柱密度(column density): 光球单位面积外面的原子数.

f值(f-value)/振子强度(oscillator strength): 从相同初态跃迁到不同终态的相对概率.

生长曲线: 自变量为能发生某跃迁的原子的柱密度(对数), 因变量为此跃迁产生的谱线的等值宽度(对数).

    \chapter{恒星内部\\The Interiors of Stars}

流体静力学平衡: 流体微元所受合力处处为零.

Kelvin--Helmholtz时标: 太阳总机械能和太阳光度之比,
\begin{equation*}
    t_\text{KH}=\frac{\frac{3}{10}\frac{GM_\odot^2}{R_\odot}}{L_\odot}.
\end{equation*}

Vogt--Russell定理: 质量和化学组成唯一决定.

Eddington光度极限: 辐射光度最大值, 超过必有质量损失.

补充内容:

流体静力学平衡方程(hydrostatic equilibrium equation):
\begin{equation*}
    \frac{\mathrm{d} P}{\mathrm{d} r}
    = -G\frac{M_r\rho}{r^2},
\end{equation*}
即$-GM_r(\rho\,\mathrm{d}A\mathrm{d}r)/r^2=\Delta P\mathrm{d}A$.

质量守恒方程(mass conservation equation):
\begin{equation*}
    \frac{\mathrm{d} M_r}{\mathrm{d} r}
    = 4\pi r^2\rho,
\end{equation*}
即$\mathrm{d} M_r=\rho (4\pi r^2\mathrm{d} r)$.

理想气体的压强积分(pressure integral):
\begin{equation*}
    P=\frac{1}{3}n(\overline{\b{p}\cdot\b{v}}).
\end{equation*}
平均是数量平均. 记忆法: 压强只需考虑一个朝向固定的板, 由各向同性, 对任意方向$i$,
\begin{equation*}
    P=n\bar{p}_i\bar{v}_i.
\end{equation*}
而$P\Delta S = \Delta p_i/\Delta t$, $P\Delta S \Delta t= \Delta p_i$, $P\Delta S (\bar{v}_i\Delta t)= P\Delta V$, $\Delta p_i \bar{v}_i = \Delta N \bar{p}_i\bar{v}_i$, $P=(\Delta P/\Delta V)\bar{p}_i\bar{v}_i=n\bar{p}_i\bar{v}_i$. 另一种记忆法是量纲法: 因为是状态量, 所以$P$与$n$, $\bar{p}_i$, $\bar{v}_i$有关, 已知$P=nkT$, 由能量均分定理知$kT$与$\bar{p}_i\bar{v}_i$同量纲(非相对论情形下甚至相同), 所以蒙$P=n\bar{p}_i\bar{v}_i$.

平均分子量(mean modular weight)~$\mu:=\bar{m}/m_\text{H}:=(\rho/n)/m_\text{H}$.

完全中性气体, $A_j:=m_j/m_\text{H}$(相当于原子核内重子数),
\begin{equation*}
    \mu_\text{n}=\frac{\sum_j N_j A_j}{\sum_j N_j}.
\end{equation*}

完全电离气体, 原子$j$的电子数$z_j$,
\begin{equation*}
    \mu_\text{i}\simeq\frac{\sum_j N_j A_j}{\sum_j N_j(1+z_j)}.
\end{equation*}
也就是说电子质量不管了.

强行假设恒星密度均匀, 蒙
\begin{equation*}
    E\sim-\frac{3}{10}\frac{GM^2}{R^2}.
\end{equation*}

Kelvin--Helmholtz时标.

核时标.

反应截面(cross section)~$\sigma(E)$. 理解1: 入射粒子有特定能量时, 单位时间单个目标粒子的反应数, 比上单位时间目标粒子附近单位面积通过的入射粒子数. 理解2: 入射粒子有特定能量时, 目标粒子附近的一块面积, 当且仅当通过这块面积的入射粒子和目标粒子反应.

反应率(reaction rate)~$r_{ix}$: 单位时间单位体积的反应数.

Gamow峰: 使得$\mathrm{d}r_{ix}/\mathrm{d}E$最大的能量. 严格说来不含缓变函数$S(E)$的影响.

电子屏蔽(electron screening): 自由电子包在原子核周围, 使之有效电荷减小, 势阱降低, 反应更容易发生.

产能率(energy generation rate)~$\epsilon_{ix}$: 单位时间单位质量物体反应释放的能量.

光度梯度方程(luminosity gradient equation)/能量守恒方程(energy conservation equation)\footnote{前面书上的名称, 后面老师视频里的名称.}:
\begin{equation*}
    \frac{\mathrm{d} L_r}{\mathrm{d} r}
    = 4\pi r^2\rho\epsilon,
\end{equation*}
即$\mathrm{d} L_r = \epsilon\,\mathrm{d} M_r= \epsilon[\rho(4\pi r^2\mathrm{d} r)]$.

电荷守恒, 核子守恒, 轻子守恒(正轻子记$+1$, 反轻子记$-1$).

$T_n$: 以$10^n\text{K}$为单位. 乱七八糟的量都约为$1$.

辐射温度梯度:
\begin{equation*}
    \begin{cases}
        \frac{\mathrm{d}P}{\mathrm{d}\tau}=\frac{F}{c}, \\
        \mathrm{d}\tau = -\bar{\kappa}\rho\,\mathrm{d}r, \\
         P=\frac{aT^4}{3}, \\
         F=\frac{L_r}{4\pi r^2}. \\
    \end{cases}
\end{equation*}

压强标高(pressure scale height)~$H_P$: $P=P_0e^{-r/H_P}$.

理想气体, $C_P-C_V$~``$=$''~$1$, $C_V$~``$=$''~$(\text{自由度})/2$. $\gamma=C_P/C_V$. 有电离, 不论定什么过程, 能量都大部分用于电离, 所以$C_P$和$C_V$巨大, $\gamma$约等于$1$.

理想气体, 绝热过程, $PV^\gamma=\text{const}$.

声速$\sqrt{\partial P/\partial \rho}$~(振动, $P$和$\rho$是力学强度量, 量纲), $P
\rho^{-\gamma}=\text{const}$.

绝热温度梯度:
\begin{equation*}
    \begin{cases}
        P\propto \rho T, \\
        P\rho^{-\gamma} \propto 1. \\
    \end{cases}
\end{equation*}

超绝热(superadiabatic): 实际温度梯度大于绝热温度梯度. 超绝热等价于对流. 对流时, 超绝热但接近绝热, 绝热温度梯度公式适用.

温度梯度大的不容易发生. 大恒星中心, 产能率高, 光度大, 对流. 小恒星边缘, 不透明度大, 且正电离, $\gamma$约等于$1$, 对流.

混合长理论(mixed-length theory): 假设泡泡能运动和压强标高$H_P$相当的距离$\ell$, 令$\ell=\alpha H_P$; $\delta(\mathrm{d}T/\mathrm{d}r)$是实际温度梯度(大小)减绝热温度梯度(大小); $f$是单位体积泡泡所受合力, 即浮力减重力; 假设泡泡平均速度$\bar{v}$满足$\rho\bar{v}^2/2=\beta\langle f\rangle\ell$, 其中$0<\beta<1$; 最后得出对流流量$F=L_r/4\pi r^2$的表达式.

Vogt--Russell定理: 质量和化学组成唯一决定.

多方模型(polytropic model), $\gamma:=(n+1)/n$, $\rho:=\rho|_{r=0}(D_n)^n$, $r:=\lambda_n\xi$, Lane--Emden方程. $P|_{r=R}=0$, $(\mathrm{d}P/\mathrm{d}r)|_{r=0}=0$. $0\le n \le 5$\footnote{
    $n=5$时半径无限, 但质量有限. $n>5$时质量无限.
}. 假设各处气体压和辐射压比例相同, 可得Eddington标准模型$n=3$.

Eddington极限: $\mathrm{d}P/\mathrm{d}r=\mathrm{d}P_\text{rad}/\mathrm{d}r$. 对大质量恒星, $\bar{\kappa}$主要由电子屏蔽导致, 可用上一章的公式.

    \chapter{太阳\\The Sun}

\begin{itemize}
    \item 光球(photosphere): $500\text{nm}$光深$1$处往下100km, 上到温度最小处.
    \item 色球(chromosphere): 温度最小处到过渡区.
    \item 过渡区(transition region): 温度激增区.
    \item 日冕(corona): 过渡区外.
\end{itemize}

较差自转(differential rotation): 低纬快, 高纬慢; 辐射区速度一样, 到差旋层(tachocline)出现速度差且随半径迅速增大, 对流区速度差一样.

Alfv\'en波: 等离子体, 磁感线像弦一样抖动, 波沿磁感应强度方向传播.

补充内容:

MSW效应\footnote{英文: effect}: 中微子转变

光球(photosphere): $500\text{nm}$光深$1$处往下100km, 上到温度最小处. 产生吸收线. 线心\underline{波长}$\!\!$处, 物质不透明度大(不透明度随波长变化), 光深等则深度小.

米粒组织(granulation): 光球底, 亮上浮暗下沉, 对流的结果.

色球(chromosphere): 温度最小处到过渡区. 产生发射线.

闪谱(flash spectrum): 日全食时看到的色球谱.

超米粒组织(supergranulation): 比米粒组织大得多.

针状体(spicule): 丝状.

过渡区(transition region): 温度激增区. 紫外观测.

日冕(corona): 禁戒线, 射电发射(韧致辐射(自由---自由发射), 同步辐射\footnote{
    相对论性电子被磁场加速.
}), X射线发射(高电离原子的发射). 非LTE, 温度无统一定义.
\begin{itemize}
    \item K日冕: 光球辐射被自由电子散射, 产生连续白光谱. F日冕内.
    \item F日冕: 光球辐射被尘埃散射. K日冕内.
    \item E日冕: 高度电离原子产生发射线. 与K日冕和F日冕能重合.
\end{itemize}

冕洞(coronal hole): 暗的, 冷的区域. X射线暗区.

太阳风(solar wind):
\begin{itemize}
    \item 快速太阳风: 电子和粒子流.
    \item 慢速太阳风: 速度快速风一半, 从封闭磁场出来.
\end{itemize}

彗尾:
\begin{itemize}
    \item 离子尾: 太阳风.
    \item 尘埃尾: 辐射压. 弯曲原因: 轨道速度不同.
\end{itemize}

北极光(aurora borealis)和南极光(aurora australis): 粒子被地磁场捕获.

Van Allen辐射带: 被地球捕获的太阳粒子在地磁南北两极间来回运动.

X射线亮区(X-ray bright region): 磁力线闭合.

Parker风模型: 等离子体(plasma), 等温, 流体静力学平衡, 理想气体, 计算得到的无穷远处数密度和压强过大. 修正, 不流体静力学平衡, $\mathrm{d}v/\mathrm{d}t=(\mathrm{d}v/\mathrm{d}r)(\mathrm{d}r/\mathrm{d}t)=v(\mathrm{d}v/\mathrm{d}r)$,
\begin{equation*}
    \begin{cases}
        P = 2\frac{\rho}{m_\text{质子}}kT, \\
        \rho v \frac{\mathrm{d}v}{\mathrm{d}r}=-\frac{\mathrm{d}P}{\mathrm{d}r}-G\frac{M_r\rho}{r^2}, \\
        \frac{\mathrm{d}M_r}{\mathrm{d}r}=4\pi r^2\rho, \\
        \frac{\mathrm{d}(4\pi r^2\rho v)}{\mathrm{d}r}=0. \\
    \end{cases}
\end{equation*}
$F=(1/2)\rho v_\text{粒子}^2v_\text{声}$. 若$4\pi r^2F$不变, $v_\text{声}$不变, 则$v_\text{粒子}$大增, 迅速超声速(supersonic), 产生激波(shock wave).

Alfv\'en波: $u=P=B^2/2\mu_0$, $v=B/\sqrt{\mu_0\rho}$.

黑子(sunspot): 光球, 本影(umbra)和半影(penumbra).

?光斑(facula): 光球, 延伸到色球成谱斑.

?谱斑(plage/flocculus[?]): 色球, 黑子旁边, H$\mathrm{\alpha}$辐射, 比周围密度高, 磁场产物.

?耀斑(flare): 色球, 磁重联(magnetic reconnection), 核反应: 分裂反应(spallation reaction), 重核子变成轻核子.

日珥(solar prominence): 色球,
\begin{itemize}
    \item 宁静日珥(quiescent prominence): 持续周到月.
    \item 暴发/活跃日珥(eruptive/active prominence): 持续数小时, 可能由宁静日珥转化来.
\end{itemize}

日冕,
\begin{itemize}
    \item 宁静日冕(quiet corona): 太阳活动弱的时候, 更集中在赤道区域.
    \item 活跃日冕(active corona): 太阳活动强的时候, 形状更复杂.
\end{itemize}

耀星(flare star): M型随机快速亮度起伏星.

    \chapter{星际介质和恒星形成\\The Interstellar Medium and Star Formation}

星际消光(interstellar extinction):
\begin{equation*}
    m_\lambda=M_\lambda+5\log_{10}d-5+A_\lambda.
\end{equation*}
$I_\lambda/I_{\lambda,0}=e^{-\tau_\lambda}\Rightarrow A_\lambda=m_\lambda-m_{\lambda,0}=-2.5\log_{10}(I_\lambda/I_{\lambda,0})=-2.5\log_{10}(e)\,\tau_\lambda$.

色余(color excess) $E(B-V)=(B-V)_\text{observed}-(B-V)_\text{intrinsic}=A_B-A_V$\footnote{这是老师的定义, 书上的定义和老师是反过来的呜呜呜\dots}.

21cm谱: 基态氢, 自旋---自旋耦合\footnote{什么是耦合? Hamilton算符中同时出现质子自旋算符和电子自旋算符就是耦合.}(超精细结构). 可测原因一: 数密度小, 电子不因碰撞跑到其他能级, 长时间后能跃迁. 可测原因二: 尺度大, 柱密度大(即原子数多), 微小辐射可以大量叠加.

Jeans判据: 孤立云, 云的质量大于Jeans质量, 即云的半径大于Jeans半径, 则塌缩.

林忠四郎\footnote{姓林, 名忠四郎.}轨迹(Hayashi track): 完全对流时流体静力学平衡的临界条件. 要在左边, 不能在右边, 在右边的要收缩. 原恒星演化自由落体阶段就在右边, 但形成原恒星后就只能沿轨迹向下.

补充内容:

柱密度(column density) $N_{d}$ : 数密度对路程积分, 单位截面的粒子数. 由第九章各种定义, $\tau_\lambda=\sigma_\lambda N_{d}$.

示踪体(tracer): 假设比例相同.

ISM加热: 宇宙线(带电粒子), 紫外光电离碳原子等各种原因, SN或星风的激波.

ISM冷却: 红外光子发射.

尘埃来源: 冷恒星包层, SN爆炸和星风, 尘埃生长(coagulation)和吸积(accretion)

Jeans判据(criterion): 维里定理, 动能理想气体动能, 势能第十章$-3/5$倍公式, 假设密度均匀, 得Jeans质量和Jeans长度.

Bonner-Ebert质量: 考虑外压强的判据.

自相似塌缩(homologous collapse): 云自由落体, 无压强外推, 等温, 可得自由落体塌缩时间.

自内向外塌缩(inside-out collapse): 密度高的中心密度增加更快.

碎裂(fragmentation): 不等温, 因为云不透明, 辐射加热云.

双极扩散(ambipolar diffusion): 当中性粒子试图穿越磁力线时, 它们与``冻结''的离子发生碰撞, 中性粒子的运动受到抑制. 然而, 如果由于重力的作用, 中性粒子的运动有一个明确的净方向, 它们仍然倾向于在这个方向上缓慢地移动. 这种缓慢的迁移过程被称为双极扩散.

原恒星演化(protostellar evolution)($1M_\odot$): 
\begin{enumerate}
    \item 自由落体阶段, 透明$\Longrightarrow$等温.
    \item 不透明度增加$\Longrightarrow$绝热坍缩$\Longrightarrow$~$T\nearrow$, 接近流体静力平衡, 原恒星(protostar).
    \item $T\sim1000\text{K}$, 尘埃蒸发, 不透明度下降, $T\nearrow$.
    \item $T\sim2000\text{K}$, 分子瓦解, 吸收能量$\Longrightarrow$核再次坍塌$\Longrightarrow$~$T\nearrow$, 氘点燃.
\end{enumerate}
覆盖在核外的材料不断落在核上, 形成激波并加热核.

主序前演化(Pre-main-sequence evolution):
\begin{itemize}
    \item $1M_\odot$\begin{enumerate}
        \item 第一百万年, 完全对流, D燃烧.
        \item $T\nearrow$, 电离, 辐射核.
        \item 核反应(pp和CNO)开始.
        \item CNO产能率高$\Longrightarrow$高温度梯度$\Longrightarrow$对流, 核心膨胀, $T\searrow$, $L\searrow$.
        \item 核反应产能远大于引力势能产能, 恒星稳定.
    \end{enumerate}
    \item $<0.5M_\odot$, C不燃烧.
    \item $<0.072M_\odot$, 无H燃烧, 不可持续, 失败.
\end{itemize}
褐矮星(brown dwarf): $>0.06M_\odot$, Li燃烧; $>0.013M_\odot$, D燃烧. 光谱型L和T.

大质量恒星, 温度高, CNO, 离开林忠四郎线横向演化, 到主序仍然有对流核.

零龄主序(zero-age main sequence, ZAMS): 刚到主序, 开始稳定H燃烧.

初始质量函数(initial mass function): 横轴质量, 纵轴单位质量单位体积(或单位面积)恒星产生数.

电离氢区(H II region), 外面是中性氢区(H I region), Str\"omgren半径: 单位体积单位时间复合数$\alpha n_\text{e}n_\text{H}$, $n_\text{e}\simeq n_\text{H}$, 恒星每秒辐射的能电离氢的光子数$N$.

大质量恒星, 星风, 辐射, 吹散云, 阻止其他恒星形成(大质量恒星形成时间短).

OB星协(association): 一堆OB星(不如星团多)一起诞生, 不能引力束缚.

T Tauri星: 小质量主序前恒星, 光度以数日为尺度大快速不规律动, 光谱特殊, 有P Cygni轮廓: 中心波长处大发射线, 但小于中心波长处小吸收, 原因是有壳膨胀, 但也有T Tauri星有反P Cygni轮廓, 甚至来回切换.

FU Orionis星: 一种T Tauri星, 质量吸积率大增, 内盘比恒星亮, 强星风.

Herbig Ae/Be星: A/B星, 强发射线(所以叫Ae/Be).

Herbig-Haro天体: 两端极长狭窄喷流. 

原行星盘(proplyds): 可能形成行星的盘.

    \chapter{主序和主序后恒星演化\\Main Sequence and Post-Main-Sequence Stellar Evolution}

原恒星演化(protostellar evolution), $1M_\odot$:
\begin{enumerate}
    \item 自由下落塌缩(free-fall collapse). 光学薄, 等温塌缩(isothermal collapse). 中心密度稍大, 塌缩速率更大, 密度差更大, 塌缩速率差更大\dots, 自内向外塌缩(inside-out collapse).
    \item 核心光学厚, 绝热塌缩(adiabatic collapse), 塌缩速率减慢, 接近流体静力学平衡, 核心形成原恒星. 核外自由下落, 产生激波\footnote{因为下落速度大于声速.}, 加热核.
    \item 尘埃蒸发, 核不透明度下降.
    \item $\text{H}_2$瓦解, 吸收能量, 核再次塌缩.
    \item 核重新流体静力学平衡, 持续吸积, 至$\text{D}$点燃.
    \item $\text{D}$燃尽, 光度剧降且温度缓降, 进入主序前.
\end{enumerate}

主序前演化(pre-main-sequence evolution):
\begin{itemize}
    \item $1M_\odot$:
    \begin{enumerate}
        \item 表面$\text{H}^-$高不透明度$\Longrightarrow$完全对流. 星体塌缩.
        \item 核温度升高$\Longrightarrow$电离$\Longrightarrow$不透明度下降$\Longrightarrow$辐射核.
        \item 核反应(pp和CNO)开始.
        \item CNO温度依赖$\Longrightarrow$高产能率$\Longrightarrow$高温度梯度$\Longrightarrow$对流核.  高产能率$\Longrightarrow$核膨胀$\Longrightarrow$温度下降且光度下降.
        \item $\text{C}$燃尽, 恒星稳定, 进入主序.
    \end{enumerate}
    \item $<0.5M_\odot$, 无$\text{C}$燃烧.
    \item $<0.072M_\odot$, 无$\text{H}$燃烧, 不可持续, 失败.
\end{itemize}

褐矮星(brown dwarf): $>0.06M_\odot$, $\text{Li}$燃烧; $>0.013M_\odot$, $\text{D}$燃烧. 光谱型L和T.

零龄主序(zero-age main sequence, ZAMS): 刚到主序, 开始稳定$\text{H}$燃烧. $>1.2M_\odot$, CNO, 对流核\footnote{这能使得核内成分保持均匀.}; $0.3M_\odot\sim1.2M_\odot$, pp, 辐射核; $<0.3M_\odot$\footnote{书上只说``最低质量'', 我猜是$<0.3M_\odot$都行.}, pp, 对流核.

小质量主序演化(low-mass main-sequence evolution), $1M_\odot$: 光度, 半径, 温度稳定增加, $\text{H}\to\text{He}$, 平均分子量增加, 核塌缩, 反应率增加.

对流超射(convective overshooting): 对流泡泡因惯性冲出对流区.

中质量主序演化(intermediate-mass main-sequence evolution), $5M_\odot$: 核稍稍损失质量, 形成小成分梯度, 末期星体收缩, 光度和温度增加.

$>10M_\odot$, 对流核消失.

主序后演化(post-main-sequence evolution): $\text{H}$燃烧止, 亚巨星支(subgiant branch), 红巨星支(red giant branch), 红巨星支上端\footnote{很火的TRGB!}(red giant tip), 蓝向/红向\footnote{这里的``蓝向/红向''是我计几取的名儿\dots}水平支(blueward/redward portion of horizontal branch), 早期/热脉冲渐近巨星支(early/thermal-pulse asymptotic giant branch), 渐近巨星后支(post-asymptotic giant branch). 挖掘(dredge-up). 超星风(superwind).

$1M_\odot$:
\begin{enumerate}
    \item 核收缩, 厚$\text{H}$燃烧壳, 光度增加, 半径稍增, 温度下降.
    \item 亚巨星支, 核迅速收缩, 星体膨胀, 温度下降, 光度增加.
    \item 温度下降$\Longrightarrow$ $\text{H}$增加$\Longrightarrow$光球不透明度增加$\Longrightarrow$表面对流. 
    \item 对流层深入中心, 红巨星支, 光度增加, 第一次挖掘. $\text{Li}$燃尽, CNO, 光球化学成分变化, 红巨星支上端.
    \item 核收缩, 强电子简并, 压强不依赖于温度. 3$\mathrm{\alpha}$反应开始, 温度增加, 压强不变, 温度不下降,  3$\mathrm{\alpha}$反应高温度依赖, 氦闪(Helium core flash).
    \item 3$\mathrm{\alpha}$反应, $\text{He}$燃烧, 核膨胀, $\text{H}$燃烧壳膨胀(被外推), 产能率下降, 光度下降, 壳收缩, 温度升高, 蓝向水平支. 壳收缩最终导致产能率再升高.
    \item 核平均分子量增加, 至核收缩, 壳膨胀冷却, 红向水平支.
    \item $\text{He}$壳燃烧, 早期渐近巨星支, 壳膨胀, 温度下降, 对流层二次深入中心(原因同第一次挖掘), 第二次挖掘.
    \item 热脉冲渐近巨星支, $\text{H}$燃烧层生成$\text{He}$落入$\text{He}$层, $\text{He}$层部分简并, 氦壳闪(Helium core flash), $\text{H}$燃烧层外推, 冷却停止燃烧, 氦壳闪终止, $\text{H}$燃烧层恢复, 循环. 光度增加, 温度降低.
    \item 快速质量损失, 末期超星风, 后渐近巨星支.
\end{enumerate}

$5M_\odot$:
\begin{enumerate}
    \item 整星KH时标收缩, 光度稍增, 半径下降, 温度升高.
    \item 厚$\text{H}$燃烧壳, 壳稍膨胀, 光度下降, 温度下降.
    \item 亚巨星支, 核迅速收缩, 星体膨胀, 温度下降, 光度增加但最后稍降.
    \item 温度下降$\Longrightarrow$ $\text{H}$增加$\Longrightarrow$光球不透明度增加$\Longrightarrow$表面对流.
    \item 对流层深入中心, 红巨星支, 光度增加, 第一次挖掘. $\text{Li}$燃尽, CNO, 光球化学成分变化, 红巨星支上端.
    \item 3$\mathrm{\alpha}$反应, $\text{He}$燃烧, 核膨胀, $\text{H}$燃烧壳膨胀(被外推), 产能率下降, 光度下降, 壳收缩, 温度升高, 蓝向水平支.  壳收缩最终导致产能率再升高.
    \item 核平均分子量增加, 至核收缩, 壳膨胀冷却, 红向水平支.
    \item $\text{He}$壳燃烧, 早期渐近巨星支, 壳膨胀, 温度下降, 对流层二次深入中心(原因同第一次挖掘), 第二次挖掘.
    \item 热脉冲渐近巨星支, 氦壳闪. 光度增加, 温度降低.
    \item 对流层三次深入中心, 第三次挖掘.
    \item 快速质量损失, 末期超星风, 后渐近巨星支.
\end{enumerate}

脉泽(maser): 微波受激辐射.

星族(stellar population): 年龄, 化学组成, 空间分布和运动特性. 星族I: 年轻, 富金属, 银盘. 星族II: 年老, 贫金属, 银晕.

球状星团(globular cluster), $n\times10^3\uparrow$, 星族II. 疏散星团(open cluster), $n\times10^3\downarrow$, 星族I.

赫氏空隙(Hertzsprung gap): 星团赫罗图空隙. 原因: 主序后快速演化.

    \chapter{恒星脉动\\Stellar Pulsation}

经典造父变星(classical Cepheid), 星族I.

不稳定带(instability strip): 赫罗图上接近垂直的一块区域, 部分脉动变星(天琴座RR变星, 造父变星, 室女W型变星, 鲸鱼ZZ型变星, 金牛RV型变星, 盾牌$\mathrm{δ}$型变星, 凤凰SX型变星和快速震荡Ap星)分布的区域, 底部在赫氏空隙.

$\mathrm{\kappa}$机制: 部分电离区(partial ionization zone), 收缩时不透明度增加, \dots$\to$膨胀$\to$透明$\to$不吸热$\to$冷$\to$收缩$\to$不透明$\to$吸热$\to$热$\to$\dots.

$\mathrm{\gamma}$机制: 温度变化比临近区域小.

太阳5分钟震荡: 太阳表面气体3--8分钟周期的一种起伏运动.

    \chapter{大质量恒星的命运\\The Fate of Massive Stars}

红超巨星(red supergiant star, RSG).

蓝超巨星(blue supergiant star, BSG).

Wolf--Rayet星(WR): 强宽发射线O型星. 氮序Wolf--Rayet星(WN): He, N发射线. 碳序Wolf--Rayet星(WC): He, C发射线.

亮蓝变星(luminous bright variable, LBV): 火药桶.

大质量恒星主序后演化\footnote{``O''代表O型主序星, ``Of''代表有明显发射线的O星超巨星.}:
\begin{itemize}
    \item $10M_\odot<M<20M_\odot$: O $\to$ RSG $\to$ BSG $\to$ SN,
    \item $20M_\odot<M<25M_\odot$: O $\to$ RSG $\to$ WN $\to$ SN,
    \item $25M_\odot<M<40M_\odot$: O $\to$ RSG $\to$ WN $\to$ WC $\to$ SN,
    \item $40M_\odot<M<85M_\odot$: O $\to$ Of $\to$ WN $\to$ WC $\to$ SN,
    \item $85M_\odot<M$: O $\to$ Of $\to$ LBV $\to$ WN $\to$ WC $\to$ SN.
\end{itemize}
可总结出两条规则:
\begin{itemize}
    \item 质量小的变红再变蓝, 质量大的一直蓝.
    \item 质量最小的不吹出He就炸, 质量次小的吹出He露出C然后炸, 质量足够的吹出He和C露出N然后炸.
\end{itemize}

超新星(supernova, SN): 爆炸.
\begin{itemize}
    \item 没H: SN I,
    \begin{itemize}
        \item 有Si: SN Ia,
        \item 没Si,
        \begin{itemize}
            \item 富He: SN Ib,
            \item 贫He: SN Ic,
        \end{itemize}
    \end{itemize}
    \item 有H: SN II,
    \begin{itemize}
        \item 随着时间发展演变出较强He特征谱线: SN IIb,
        \item 光度下降时仍然显示较强H特征谱线: SN IIn (``正常'' SN II),
        \begin{itemize}
            \item 光变曲线极大后线性下降(``linear''): SN II-L,
            \item 光变曲线极大后有个平台(``plateau''): SN II-P.
        \end{itemize}
    \end{itemize}
\end{itemize}

$\mathrm{\gamma}$射线暴(gamma-ray burst, GRB), $<2\,\text{s}$短暴, 硬(高能), 可能源于中子星--中子星/黑洞并合, $>2\,\text{s}$长暴, 软(低能), 可能源于超新星.

\end{document}
