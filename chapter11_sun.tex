\chapter{太阳\\The Sun}

MSW效应\footnote{英文: effect}: 中微子转变

光球(photosphere): $500\text{nm}$光深$1$处往下100km, 上到温度最小处. 产生吸收线. 线心\underline{波长}$\!\!$处, 物质不透明度大(不透明度随波长变化), 光深等则深度小.

米粒组织(granulation): 光球底, 亮上浮暗下沉, 对流的结果.

色球(chromosphere): 光球上到过渡区. 产生发射线.

闪谱(flash spectrum): 日全食时看到的色球谱.

超米粒组织(supergranulation): 比米粒组织大得多.

针状体(spicule): 丝状.

过渡区(transition region): 温度激增区. 紫外观测.

日冕(corona): 禁戒线, 射电发射(韧致辐射(自由---自由发射), 同步辐射\footnote{
    相对论性电子被磁场加速.
}), X射线发射(高电离原子的发射). 非LTE, 温度无统一定义.
\begin{itemize}
    \item K日冕: 光球辐射被自由电子散射, 产生连续白光谱. F日冕内.
    \item F日冕: 光球辐射被尘埃散射. K日冕内.
    \item E日冕: 高度电离原子产生发射线. 与K日冕和F日冕能重合.
\end{itemize}

冕洞(coronal hole): 暗的, 冷的区域. X射线暗区.

太阳风(solar wind):
\begin{itemize}
    \item 快速太阳风: 电子和粒子流.
    \item 慢速太阳风: 速度快速风一半, 从封闭磁场出来.
\end{itemize}

彗尾:
\begin{itemize}
    \item 离子尾: 太阳风.
    \item 尘埃尾: 辐射压. 弯曲原因: 轨道速度不同.
\end{itemize}

北极光(aurora borealis)和南极光(aurora australis): 粒子被地磁场捕获.

Van Allen辐射带: 被地球捕获的太阳粒子在地磁南北两极间来回运动.

X射线亮区(X-ray bright region): 磁力线闭合.

Parker风模型: 等离子体(plasma), 等温, 流体静力学平衡, 理想气体, 计算得到的无穷远处数密度和压强过大. 修正, 不流体静力学平衡, $\mathrm{d}v/\mathrm{d}t=(\mathrm{d}v/\mathrm{d}r)(\mathrm{d}r/\mathrm{d}t)=v(\mathrm{d}v/\mathrm{d}r)$,
\begin{equation*}
    \begin{cases}
        P = 2\frac{\rho}{m_\text{质子}}kT, \\
        \rho v \frac{\mathrm{d}v}{\mathrm{d}r}=-\frac{\mathrm{d}P}{\mathrm{d}r}-G\frac{M_r\rho}{r^2}, \\
        \frac{\mathrm{d}M_r}{\mathrm{d}r}=4\pi r^2\rho, \\
        \frac{\mathrm{d}(4\pi r^2\rho v)}{\mathrm{d}r}=0. \\
    \end{cases}
\end{equation*}
$F=(1/2)\rho v_\text{粒子}^2v_\text{声}$. 若$4\pi r^2F$不变, $v_\text{声}$不变, 则$v_\text{粒子}$大增, 迅速超声速(supersonic), 产生激波(shock wave).

Alfv\'en波: $u=P=B^2/2\mu_0$, $v=B/\sqrt{\mu_0\rho}$.

黑子(sunspot): 光球, 本影(umbra)和半影(penumbra).

?光斑(facula): 光球, 延伸到色球成谱斑.

?谱斑(plage/flocculus[?]): 色球, 黑子旁边, H$\mathrm{\alpha}$辐射, 比周围密度高, 磁场产物.

?耀斑(flare): 色球, 磁重联(magnetic reconnection), 核反应: 分裂反应(spallation reaction), 重核子变成轻核子.

日珥(solar prominence): 色球,
\begin{itemize}
    \item 宁静日珥(quiescent prominence): 持续周到月.
    \item 暴发/活跃日珥(eruptive/active prominence): 持续数小时, 可能由宁静日珥转化来.
\end{itemize}

\begin{itemize}
    \item 宁静日冕(quiet corona): 太阳活动弱的时候, 更集中在赤道区域.
    \item 活跃日冕(active corona): 太阳活动强的时候, 形状更复杂.
\end{itemize}

磁发电机理论(magnetic dynamo theory), 看老师视频去.

耀星(flare star): M型随机快速亮度起伏星.
