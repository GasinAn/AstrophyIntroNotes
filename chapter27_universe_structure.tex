\chapter{宇宙结构\\The Structure of the Universe}

宇观尺度: $>3\times10^{8}\,\text{ly}$.

宇宙学原理: 时空在宇观尺度下是空间均匀且各项同性的.

宇宙静系: 认为时空在宇观尺度下空间均匀且各项同性的所有观者组成的参考系. 宇宙静系被认为是唯一的.

共动(cosmoving)坐标系: 使得所有宇宙静系观者坐标不变的坐标系.

固有距离: ``真实距离''. 共动距离: 共动坐标系中的``坐标差''.

Hubble定律: 星系退行速度和星系固有距离成正比. 因为在宇宙静系中, 宇宙静系观者的退行速度和其固有距离严格成正比, 所以这般表述的Hubble定律在统计意义上是绝对成立的.\footnote{如果红移非常大, 测量距离和红移后能修正距离得到固有距离, 但得到的是那时刻的固有距离和那时刻的红移, 比例系数(Hubble常数)不同, 这样Hubble定律观测上就没法成立了.} 若星系固有距离足够大, 星系退行速度是不是超光速啦?\footnote{答案请看梁师爷书第372页.}

Hubble参数: 宇宙静系观者的退行速度和其固有距离的比. Hubble常数: 今天的Hubble参数. 天文学家说Hubble常数, 即非常数, 是名常数.

空间膨胀/收缩, 即宇宙静系观者的固有距离变大/变小, 即共动坐标系下的度规分量(绝对值)变大/变小.

退行速度(recessional velocity): 星系处的宇宙静系观者的速度, 或者说知道距离后用Hubble定律算出的径向速度, 或者说``空间本身膨胀/收缩的速度''. 本动速度(peculiar velocity): 星系相对于星系处的宇宙静系观者的速度, 或者说星系速度和星系退行速度的差, 或者说``星系相对于空间运动的速度''. Hubble流: 脑壳里设想的, 星系因退行速度而有的运动, 或者说星系跟随空间本身的膨胀/收缩而产生的运动.

示距天体(distance indicator): 可以用来测距的东西.

宇宙微波背景辐射(cosmic microwave background, CMB): 大爆炸后温度下降到一定值, 原子核与电子结合呈中性原子, 光子几乎不与中性原子作用, 因此发生光子退耦(decoupling), 即光子的平均自由程变得极大\footnote{除以光速后远大于宇宙当今年龄.}, 自由传播, 产生几乎各项同性的微波辐射. 已知退耦时温度大约$3000\text{K}$, 为何现在CMB温度大约$3\text{K}$?\footnote{相信不用看师爷书第388页也能正确回答这个问题呢$\!\sim$}
