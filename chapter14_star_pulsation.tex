\chapter{恒星脉动\\Stellar Pulsation}

经典造父变星(classical Cepheid), 星族I.

不稳定带(instability strip): 赫罗图上接近垂直的一块区域, 部分脉动变星(天琴座RR变星, 造父变星, 室女W型变星, 鲸鱼ZZ型变星, 金牛RV型变星, 盾牌$\mathrm{\delta}$型变星, 凤凰SX型变星和快速震荡Ap星)分布的区域, 底部在赫氏空隙.

$\mathrm{\kappa}$机制: 部分电离区(partial ionization zone), 收缩时不透明度增加, \dots$\to$膨胀$\to$透明$\to$不吸热$\to$冷$\to$收缩$\to$不透明$\to$吸热$\to$热$\to$\dots.

$\mathrm{\gamma}$机制: 温度变化比临近区域小.

太阳5分钟震荡: 太阳表面气体3--8分钟周期的一种起伏运动.
