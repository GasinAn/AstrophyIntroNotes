\chapter{星系特性\\The Nature of Galaxies}

Hubble序列:
\begin{itemize}
    \item 椭圆星系(elliptical): E0--E7, \underline{看起来}$\!\!$圆到扁.
    \item 透镜状星系(lenticular): S0/SB0, 盘状, 核和盘差不多大, 无旋臂.
    \item 旋涡星系(spiral): Sa--Sc, 核球状, 旋臂紧到松.
    \item 棒旋星系(barred spiral): SBa--SBc, 核棒状, 旋臂紧到松.
    \item 不规则星系(irregular): Ir, 一堆奇葩.
\end{itemize}

de Vaucouleurs轮廓: 旋涡星系核/大椭圆星系, (单位天区某频段)星等(赋以恰当零点后)正比于星系中心距离的$1/4$次方. S\'ersic轮廓: $1/n$次方.

Tully--Fisher关系: 旋涡星系, 光度--最大视向速度关系, 光度用某波段绝对星等表示, 最大视向速度测不同频率的流量来定.

Faber--Jackson关系: 椭圆星系, (某波段)光度正比于中心速度弥散的4次方.

Lin--Shu密度波理论: 莫名其妙能行的堵车论.
