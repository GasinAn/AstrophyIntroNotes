\chapter{大质量恒星的命运\\The Fate of Massive Stars}

红超巨星(red supergiant star, RSG).

蓝超巨星(blue supergiant star, BSG).

Wolf--Rayet星(WR): 强宽发射线O型星. 氮序Wolf--Rayet星(WN): He, N发射线. 碳序Wolf--Rayet星(WC): He, C发射线.

亮蓝变星(luminous bright variable, LBV): 火药桶.

大质量恒星主序后演化\footnote{``O''代表O型主序星, ``Of''代表有明显发射线的O星超巨星.}:
\begin{itemize}
    \item $10M_\odot<M<20M_\odot$: O $\to$ RSG $\to$ BSG $\to$ SN,
    \item $20M_\odot<M<25M_\odot$: O $\to$ RSG $\to$ WN $\to$ SN,
    \item $25M_\odot<M<40M_\odot$: O $\to$ RSG $\to$ WN $\to$ WC $\to$ SN,
    \item $40M_\odot<M<85M_\odot$: O $\to$ Of $\to$ WN $\to$ WC $\to$ SN,
    \item $85M_\odot<M$: O $\to$ Of $\to$ LBV $\to$ WN $\to$ WC $\to$ SN.
\end{itemize}
可总结出两条规则:
\begin{itemize}
    \item 质量小的变红再变蓝, 质量大的一直蓝.
    \item 质量最小的不吹出He就炸, 质量次小的吹出He露出C然后炸, 质量足够的吹出He和C露出N然后炸.
\end{itemize}

超新星(supernova, SN): 爆炸.
\begin{itemize}
    \item 没H: SN I,
    \begin{itemize}
        \item 有Si: SN Ia,
        \item 没Si,
        \begin{itemize}
            \item 富He: SN Ib,
            \item 贫He: SN Ic,
        \end{itemize}
    \end{itemize}
    \item 有H: SN II,
    \begin{itemize}
        \item 随着时间发展演变出较强He特征谱线: SN IIb,
        \item 光度下降时仍然显示较强H特征谱线: SN IIn (``正常'' SN II),
        \begin{itemize}
            \item 光变曲线极大后线性下降(``linear''): SN II-L,
            \item 光变曲线极大后有个平台(``plateau''): SN II-P.
        \end{itemize}
    \end{itemize}
\end{itemize}

$\mathrm{\gamma}$射线暴(gamma-ray burst, GRB), $<2\,\text{s}$短暴, 硬(高能), 可能源于中子星--中子星/黑洞并合, $>2\,\text{s}$长暴, 软(低能), 可能源于超新星.
