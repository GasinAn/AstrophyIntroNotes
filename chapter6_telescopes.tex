\chapter{望远镜\\Telescopes}

焦比(focal ratio)~$F$: 焦距与口径之比.

Airy斑\footnote{``斑的英文是disk.''}, $\theta\simeq1.22(\lambda/D)$.

主动光学(active optics): 消除由于热效应和望远镜移动时反射镜上重力变化造成的镜面变形. 每秒一次至百秒一次.

自适应光学(adaptive optics): 消除由于大气湍动造成的星像变形. 每秒数十次至每秒数百次.

补充内容:

Snell律: $n_{1\lambda}\sin\theta_1=n_{2\lambda}\sin\theta_2$, $n_\lambda:=c/v_\lambda$.

透镜靠近光轴处, lensmaker's formula:
\begin{equation*}
    \frac{1}{f_\lambda}=(n_\lambda-1)\left(\frac{1}{R_1}+\frac{1}{R_2}\right).
\end{equation*}
$f_\lambda$是凸透镜(converging lens)为正, 凹透镜(diverging lens)为负. $R$是镜面向外凸为正, 向内凹为负.

球面反射镜靠近光轴处$\left\lvert f\right\rvert=R/2$, 凹面镜(converging mirror)为正, 凸面镜(diverging mirror)为负.

近光轴, 底片比例尺(plate scale)~$\mathrm{d}\theta/\mathrm{d}y=1/f$.

衍射极限$\theta_\text{min}\simeq1.22(\lambda/D)$.

畸变(aberration):
\begin{itemize}
    \item 色差(chromatic aberration): 折射镜, 折射率不同, 焦距不一, 色散.
    \item 球差(spherical aberration): (球面镜)光不沿光轴时, 汇聚到光轴不同位置. 用抛物面镜解决.
    \item 彗差(coma): 抛物面焦距依赖于角度.
    \item 散光(astigmatism): 镜子不同部分焦点不同.
    \item 场曲(curvature of field): 设计镜子修正散光后, 焦面不是平面.
    \item 场变(distortion of field): 设计镜子修正散光后, 底片比例尺与光和光轴的距离相关, 图像变形.
\end{itemize}

照度(illumination)~$J$: 接收器单位面积单位时间接收到的能量.

$J\propto F^{-2}$.

焦点系统:
\begin{itemize}
    \item 主焦点(prime focus): 反射到中间.
    \item Newtonian: 加平面镜, 反射到旁边.
    \item Cassegrain: 主镜抛物面, 加凸面镜, 反射到主镜的洞里, 焦距增大.
    \item Ritchey-Chr\'etien: Cassegrain主镜抛物面$\to$双曲面.
    \item 折轴式(coud\'e): Cassegrain再加平面镜, 反射到旁边.
\end{itemize}

Schmidt: 球面+旋转四次曲面.

$1\,\text{Jy}=10^{-26}\,\text{W/(m}^2\!\cdot\text{Hz)}$. $1\,\text{erg}=10^{-7}\,\text{J}$.
