\chapter{密近双星系统\\Close Binary Star Systems}

平面圆型限制性三体问题: 两大星圆轨道绕转, 另有一小星. 取非惯性系使两大星静止, 则小星的等效势能(effective potential energy)为
\begin{equation*}
    U_{m,\text{eff}}=-\frac{GM_1m}{r_1}-\frac{GM_2m}{r_2}-\frac{1}{2}m(\omega r)^2,
\end{equation*}
其中$r_1$, $r_2$, $r$分别是小星与大星1, 大星2, 质心的距离. $m(\omega r)^2/2$为离心势能(centrifugal potential energy), 可以认为本来惯性系中有动能$m(\omega r)^2/2$, 非惯性系无了, 加到势能中去.

等效引力势(effective gravitational potential) $\Phi=U_{m,\text{eff}}/m$. Lagrange点: 平衡点($\nabla\Phi=0$点), 共5个, $L_1$, $L_2$, $L_3$不稳定, $L_4$, $L_5$只在两大星质量比大时稳定. 等势面(equipotential surface): 等$\Phi$面. Roche瓣: 临界等势面($L_1$所在的等势面), 8字绕两大星连线转一圈.

密近双星(close binary),
\begin{itemize}
    \item 不接双星(detached binary): 零星充满Roche瓣,
    \item 半接双星(semidetached binary): 一星充满Roche瓣,
    \item 相接双星(contact binary): 两星充满Roche瓣.
\end{itemize}

主星(primary star): 半接双星中不充满Roche瓣的, 计质量$M_1$, 次星(secondary star): 半接双星中充满Roche瓣的, 计质量$M_2$\footnote{$M_1$和$M_2$关系不定.}.

质量传输:
\begin{equation*}
    \begin{cases}
        M=\text{const},\\
        L=\mu\sqrt{GMa}=\text{const},\\
        \frac{a^3}{P^2}=\frac{GM}{4\pi^2}\\
    \end{cases}
\end{equation*}

双星演化\footnote{课本中的特例, 双中等质量恒星, 星1初始质量大于星2.}:
\begin{enumerate}
    \item 两颗小星星快乐地生活在一起.
    \item 星1充满Roche瓣, 物质流向星2并形成公共包层(common envelope), 第一次公共包层阶段, 距离减小\footnote{大质量流向小质量, 距离减小; 小质量流向大质量, 距离增大.}.
    \item 公共包层抛去, 星1白矮星, 中间阶段.
    \item 星2充满Roche瓣, 物质流向星1并再次形成公共包层, 第二次公共包层阶段, 距离再次减小.
    \item 公共包层抛去, 星2也白矮星, 引力辐射, 距离不断减小.
    \item 大半径小质量白矮星充满Roche瓣, 在小半径大质量白矮星周围形成吸积盘(accretion disk).
    \item 超过Chandrasekhar极限, 野生的Ia型超新星出现了!
\end{enumerate}

新星(nova): 亮暗亮暗. 超新星(supernova): 一次性爆炸.

Ia型超新星模型,
\begin{itemize}
    \item 双简并模型(double-degenerate model): 双白矮星, 主星吃次星, 主星内部核反应, 爆炸, 可能形成中子星.
    \item 单简并模型(single-degenerate model): 单白矮星, 吃另一正常星, 彻底爆炸干净.
\end{itemize}

毫秒脉冲星(millisecond pulsar), 吃伴星, 角动量增大加速旋转. 伴星可能是短周期扁轨道中子星或长周期圆轨道白矮星.
