\chapter{恒星光谱分类\\The Classification of Stellar Spectra}

Havard光谱分类: OBAFGKM, LT.
\begin{itemize}
    \item O型: 强He II线, He I线.
    \item B型: 强He I线(B2最强), Balmer线.
    \item A型: 强Balmer线(A0最强), Ca II线.
    \item F型: Balmer线, Ca II线, 中性金属线.
    \item G型: Ca II线, 中性金属线.
    \item K型: 金属线, 强Ca II线(K0最强\footnote{Ca II的H和K线(Fraunhofer线), 但金属线是主导的.}).
    \item M型: 分子线\footnote{分子线主导, 但金属线强如K型.}, 金属线.
\end{itemize}

Morgan--Keenan(MK)光度级: Ia-O, Ia--Ib, II--V, VI\footnote{亦记作``sd''}, D\footnote{亦记作VII型?}.
\begin{itemize}
    \item I型: 超巨星(supergiant).
    \begin{itemize}
        \item Ia-O型: 极端亮超巨星(extreme, luminous supergiant).
        \item Ia型: 亮超巨星(luminous supergiant).
        \item Ib型: 次亮超巨星(less luminous supergiant).
    \end{itemize}
    \item II型: 亮巨星(bright giant).
    \item III型: 正常巨星(normal giant).
    \item IV型: 亚巨星(subgaint).
    \item V型: 主序星(main-sequence star)/矮星(dwarf).
    \item VI型: 亚矮星(subdwarf).
    \item D型: 白矮星(white dwarf).
\end{itemize}

Boltzmann 方程:
\begin{equation*}
    P \propto g\,e^{-E/kT}.
\end{equation*}

Maxwell--Boltzmann速度分布函数: 根据Boltzmann 方程, 对于理想气体, $\mathrm{d}v_x\mathrm{d}v_y\mathrm{d}v_z$内的概率正比于$e^{-m(v_x^2+v_y^2+v_z^2)/2kT}$, 或者写成
\begin{equation*}
    p(v_x,v_y,v_z) \mathrm{d}v_x\mathrm{d}v_y\mathrm{d}v_z \propto e^{-\frac{m(v_x^2+v_y^2+v_z^2)}{2kT}} \mathrm{d}v_x\mathrm{d}v_y\mathrm{d}v_z.
\end{equation*}
Gauss积分, $e^{-\xi^2/2\sigma^2}$归一化系数是$1/\sqrt{2\pi}\sigma$, 于是乎有
\begin{equation*}
    p(v_x,v_y,v_z) \mathrm{d}v_x\mathrm{d}v_y\mathrm{d}v_z = \left(\frac{m}{2\pi kT}\right)^{3/2} e^{-\frac{m(v_x^2+v_y^2+v_z^2)}{2kT}} \mathrm{d}v_x\mathrm{d}v_y\mathrm{d}v_z.
\end{equation*}
这是速度分布, 转化到速率分布, 只需把对小方块$\mathrm{d}v_x\mathrm{d}v_y\mathrm{d}v_z$的积分变成对小球壳$4\pi v^2\mathrm{d}v$的积分, 于是乎有
\begin{equation*}
    p(v) \mathrm{d}v = \left(\frac{m}{2\pi kT}\right)^{3/2} e^{-\frac{mv^2}{2kT}} 4\pi v^2\mathrm{d}v.
\end{equation*}
变成数密度分布, 只需乘上数密度, 于是乎有
\begin{equation*}
    n_v \mathrm{d}v = n \left(\frac{m}{2\pi kT}\right)^{3/2} e^{-\frac{mv^2}{2kT}} 4\pi v^2\mathrm{d}v.
\end{equation*}
也就是说, $4\pi v^2\mathrm{d}v$是简并度$g$.

最可几速度(most probable speed): 求导.

方均根速度(root-mean-square speed): 能量均分定理$m\overline{v^2} /2=3(kT/2)$. 

Saha方程: 根据Boltzmann 方程,
\begin{equation*}
    \frac{N_{i+1}N_\text{e}}{N_{i}} = \frac{Z_{i+1}Z_\text{e}}{Z_{i}}.
\end{equation*}
其中
\begin{equation*}
    Z_{i+1} = \sum_j g_{i+1,j} e^{-E_{i+1,j}/kT},
\end{equation*}
\begin{equation*}
    Z_{i} = \sum_j g_{i,j} e^{-E_{i,j}/kT},
\end{equation*}
\begin{equation*}
    Z_\text{e} =  \int g_\text{e} e^{-[(p_x^2+p_y^2+p_z^2)/2m_\text{e}]/kT} \frac{\mathrm{d}x\mathrm{d}y\mathrm{d}z\mathrm{d}p_x\mathrm{d}p_y\mathrm{d}p_z}{h^3}.
\end{equation*}
上面三式都以束缚-自由临界状态能量为能量零点, 其中第三式运用了结论``一个微观状态等价于$xp_xyp_yzp_z$空间中体积为$h^3$的小区域''.
下面计算$Z_e$. $g_\text{e}=2$, 因此
\begin{equation*}
    Z_\text{e} = \frac{2\int\mathrm{d}x\mathrm{d}y\mathrm{d}z}{h^3} \int e^{-(p_x^2+p_y^2+p_z^2)/2m_\text{e}kT}\mathrm{d}p_x\mathrm{d}p_y\mathrm{d}p_z,
\end{equation*}
显然$V=\int\mathrm{d}x\mathrm{d}y\mathrm{d}z$, 后面是Gauss积分很好算, 结果就是
\begin{equation*}
    Z_\text{e} = \frac{2V}{h^3} (\sqrt{2\pi}\sqrt{m_\text{e}kT})^3 = 2 V \left(\frac{2\pi m_\text{e}kT}{h^2}\right)^{3/2} = 2 V \left(\frac{h}{\sqrt{2\pi m_\text{e}kT}}\right)^{-3}.
\end{equation*}
令
\begin{equation*}
    \widetilde{Z}_{i+1}  = \sum_j g_{i+1,j} e^{-(E_{i+1,j}-E_{i+1,1})/kT},
\end{equation*}
\begin{equation*}
    \widetilde{Z}_{i}  = \sum_j g_{i,j} e^{-(E_{i,j}-E_{i,1})/kT},
\end{equation*}
\begin{equation*}
    z_\text{e}  = 2 \left(\frac{2\pi m_\text{e}kT}{h^2}\right)^{3/2},
\end{equation*}
则由$\chi=E_{i+1,1}-E_{i,1}$和$n_\text{e}=N_\text{e}/V$可得
\begin{equation*}
    \frac{N_{i+1}n_\text{e}}{N_{i}} = \frac{\widetilde{Z}_{i+1}z_\text{e}}{\widetilde{Z}_{i}}e^{-\chi/kT}.
\end{equation*}
HII只有一种状态, $S=0$, 根据$S$和$Z$的关系可得$Z_\text{HII}=1$.

分光视差(spectroscopic parallax)~$d=100^{(m-M)/5}$, $m$当然可测, $M$通过看光谱比对H-R图可测, 和视差没有半毛关系!
