\chapter{双星和恒星参量\\Binary Systems and Stellar Parameters}

分类:
\begin{itemize}
    \item 光学双星(optical double): 假的.
    \item 视双星(visual binary): 都能看到, 可分辨开.
    \item 天体测量双星(astrometric binary): 一个可见, 振荡运动.
    \item 食双星(eclipsing binary): 有掩食.
    \item 光谱双星(spectrum binary): 两个堆叠的, 独立的, 可识别的光谱.
    \item 分光双星(spectroscopic binary): 谱线周期性运动.
\end{itemize}

视双星. $i$: 长轴与天球切面夹角. $\tilde{\alpha}_1$, $\tilde{\alpha}_2$: 星和自己轨道对称中心的最大角距离. $d$: 双星-地球距离. $i$是可推定的.
\begin{equation*}
    \begin{cases}
        m_1 \tilde{\alpha}_1 = m_2 \tilde{\alpha}_2, \\
        a \cos i = (\tilde{\alpha}_1+\tilde{\alpha}_2) d. \\
    \end{cases}
\end{equation*}

分光双星, $e\simeq0$. $i$: 长轴与天球切面夹角. $v_{1r}^\text{max}$, $v_{2r}^\text{max}$: 星视向速度最大值. $P$: 双星运动周期. $\left\langle \sin^3 i\right\rangle \simeq 2/3$.
\begin{equation*}
    \begin{cases}
        m_1 v_{1r}^\text{max} = m_2 v_{2r}^\text{max}, \\
        2 \pi a = [(v_{1r}^\text{max}+v_{2r}^\text{max}) / \sin i] P. \\
    \end{cases}
\end{equation*}

分光食双星, $i\simeq90^{\circ}$, $e\simeq0$, $a\gg R$. $t_\text{a}$: 亮度开始下降到主极小的时刻. $t_\text{b}$: 亮度下降到主极小的时刻. $t_\text{c}$: 亮度开始从主极小上升的时刻. $v$: 双星相对速度.
\begin{equation*}
    \begin{cases}
        2 R_\text{小} = v (t_\text{b}-t_\text{a}), \\
        2 R_\text{大} = v (t_\text{c}-t_\text{a}). \\
    \end{cases}
\end{equation*}

食双星, $i\simeq90^{\circ}$, 忽略临边昏暗. $B_\text{max}$: 亮度极大值. $B_\text{pmin}$: 亮度主极小值. $B_\text{smin}$: 亮度次极小值.
\begin{equation*}
    \begin{cases}
        B_\text{max} \propto \pi R_\text{小}^2 \sigma T_\text{小}^4 + \pi R_\text{大}^2 \sigma T_\text{大}^4, \\
        B_\text{pmin} \propto \pi R_\text{大}^2 \sigma T_\text{大}^4 \\
        B_\text{smin} \propto \pi R_\text{小}^2 \sigma T_\text{小}^4 + (\pi R_\text{大}^2 - \pi R_\text{小}^2) \sigma T_\text{大}^4. \\
    \end{cases}
\end{equation*}

搜寻地外行星方法, 懒得记了, 自己看PPT.
