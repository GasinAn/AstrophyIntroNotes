\chapter{恒星简并残骸\\The Degenerate Remnants of Stars}

白矮星(white dwarf):
\begin{itemize}
    \item DA\footnote{A型星.}: H (Balmer)线,
    \item DB\footnote{B型星.}: He I线,
    \item DC\footnote{连续谱(continuous spectrum).}: 没线,
    \item DO\footnote{O型星.}: He II线\footnote{还有弱He I线或弱H线.},
    \item DZ\footnote{金属符号``Z''.}: 金属线,
    \item DQ\footnote{没人懂得为什么用字母``Q''\dots}: C线.
\end{itemize}

Chandrasekhar极限: 白矮星最大质量.

白矮星冷却时标: 内能除以光度.

同步辐射(synchrotron radiation): 相对论性电子在弱磁场中, 连续谱, 圆轨道平面方向线偏振. 曲率辐射(curvature radiation): 相对论性电子在强磁场中, 连续谱, 磁力线平面方向线偏振.

光速圆柱面\footnote{老师称之为``光柱面'', 我用的是权威翻译$\!\sim$}(light cylinder): 以脉冲星(pulsar)自转轴为对称轴的, 半径为光速和脉冲星角速度之比的圆柱面.

磁星(magnetar): 极端强磁场中子星(neutron star), 有相对慢的自转.
