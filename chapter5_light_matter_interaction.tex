\chapter{光与物质的相互作用\\The Interaction of Light and Matter}

Kirchhoff律:
\begin{enumerate}
    \item 热致密气体或热固体产生连续谱, 无吸收线.
    \item 热弥漫气体带发射线.
    \item 冷弥漫气体在连续谱源前, 连续谱带吸收线.
\end{enumerate}

补充内容:

刻线间距$d$, 反射光与光栅法线夹角$\theta$, 光谱阶数$n$, 波长$\lambda$, $d\sin\theta=n\lambda$.

波长$\lambda$, 可分辨的最小波长差$\Delta\lambda$, 光谱阶数$n$, 刻线总数$N$, 分辨本领(resolving power)~$R=\lambda/\Delta\lambda=nN$.

Compton效应: 高能光子打低能(静止)电子, 光子波长变长. 逆Compton效应: 低能光子打高能电子, 光子波长变短. $\Delta\lambda=1-\cos\theta$.

$E_n=E_1/n^2$, $r_n=n^2r_1$. $E_1\simeq-13.6\text{eV}$, $r_1\simeq0.05\text{nm}$.

HI量子数$(n,l,m_l,m_s)$. 主量子数$n\in\mathbb{N}_+$, 轨道量子数$l=0,\dots,n-1$, 轨道磁量子数$m_l=-l,\dots,l$, 自旋磁量子数$m_s=-1/2,1/2$~($s=1/2$). 磁场方向为$z$方向, 轨道角动量$L=\sqrt{l(l+1)}\hbar$, $z$方向轨道角动量$L_z=m_l\hbar$, 自旋角动量$S=\sqrt{s(s+1)}\hbar=(\sqrt{3}/2)\hbar$, $z$方向自旋角动量$S_z=m_s\hbar$.

选择定则: $\Delta l=\pm1$, $\Delta m_l=0\,\text{或}\pm1$~($\,0\to0\,\text{禁戒}$).
