\chapter{光的连续谱\\The Continuous Spectrum of Light}

视差(parallax angle): 从天体上看, 地球和太阳的最大角距离.

$1\text{rad}\simeq 206265''$, $1\text{pc}\simeq 206265\text{AU}\simeq 3.26\text{ly}$.

视星等(apparent magnitude)~$m$. 辐射流量(radiant flux)~$F$: 仪器单位面积每秒接收到的能量. 光度(luminosity)~$L$: 天体每秒辐射的总能量.
\begin{equation*}
    m-m_0 = -2.5\lg \frac{F}{F_0}, F=100^{-(m-m_0)/5}F_0.
\end{equation*}
记忆法: 视星等五等, 亮度一百倍; 视星等越小越亮.
\begin{equation*}
    F = \frac{L}{4\pi r^2}.
\end{equation*}

绝对星等(absolute magnitude)~$M$: $10\text{pc}$处视星等. 距离模数(distance modulus)~$m-M$. 自己推距离模数公式\footnote{第二个作业!}.

Stefan-Boltzmann律: $F=\sigma T^4$.~$B(T)=\sigma T^4/\pi$: 黑体垂直于单位面元方向单位立体角内单位时间辐射的能量.
\begin{equation*}
    F = \int_{\theta\in[0,\pi]}B\cos\theta\,\mathrm{d}\Omega.
\end{equation*}
Wien位移\footnote{~``位移''的英文是displacement.}律: $\lambda_\text{max}T=(500\text{nm})(6000\text{K})$.

色指数(color index)~$X-Y$: 不同``波段''视星等的差, 等于不同``波段''绝对星等的差. 热星等(bolometric magnitude)~$m_\text{bol}$和$M_\text{bol}$: 全波段星等. 热改正(bolometric correction)~$BC=m_\text{bol}-V=M_\text{bol}-M_V$.

颜色-颜色图\footnote{不是色色图! 不是!}(color-color diagram): 横轴某色指数, 纵轴另一个. 黑体一条直线, 恒星线在黑体线下.
