\chapter{星际介质和恒星形成\\The Interstellar Medium and Star Formation}

原恒星演化和主序前演化见下一章.

星际消光(interstellar extinction):
\begin{equation*}
    m_\lambda=M_\lambda+5\log_{10}d-5+A_\lambda.
\end{equation*}
$I_\lambda/I_{\lambda,0}=e^{-\tau_\lambda}\Rightarrow A_\lambda=m_\lambda-m_{\lambda,0}=-2.5\log_{10}(I_\lambda/I_{\lambda,0})=-2.5\log_{10}(e)\,\tau_\lambda$.

色余(color excess) $E(B-V)=(B-V)_\text{observed}-(B-V)_\text{intrinsic}=A_B-A_V$\footnote{这是老师的定义, 书上的定义和老师是反过来的呜呜呜\dots}.

21cm谱: 基态氢, 自旋---自旋耦合\footnote{什么是耦合? Hamilton算符中同时出现质子自旋算符和电子自旋算符就是耦合.}(超精细结构). 可测原因一: 数密度小, 电子不因碰撞跑到其他能级, 长时间后能跃迁. 可测原因二: 尺度大, 柱密度大(即原子数多), 微小辐射可以大量叠加.

Jeans判据: 孤立云, 云的质量大于Jeans质量, 即云的半径大于Jeans半径, 则塌缩.

林忠四郎\footnote{姓林, 名忠四郎.}轨迹(Hayashi track): 完全对流时流体静力学平衡的临界条件. 要在左边, 不能在右边, 在右边的要收缩. 原恒星演化自由落体阶段就在右边, 但形成原恒星后就只能沿轨迹向下.

补充内容:

柱密度(column density) $N_{d}$ : 数密度对路程积分, 单位截面的粒子数. 由第九章各种定义, $\tau_\lambda=\sigma_\lambda N_{d}$.

示踪体(tracer): 假设比例相同.

ISM加热: 宇宙线(带电粒子), 紫外光电离碳原子等各种原因, SN或星风的激波.

ISM冷却: 红外光子发射.

尘埃来源: 冷恒星包层, SN爆炸和星风, 尘埃生长(coagulation)和吸积(accretion)

Jeans判据(criterion): 维里定理, 动能理想气体动能, 势能第十章$-3/5$倍公式, 假设密度均匀, 得Jeans质量和Jeans长度.

Bonner-Ebert质量: 考虑外压强的判据.

自相似塌缩(homologous collapse): 云自由落体, 无压强外推, 等温, 可得自由落体塌缩时间.

自内向外塌缩(inside-out collapse): 密度高的中心密度增加更快.

碎裂(fragmentation): 不等温, 因为云不透明, 辐射加热云.

褐矮星(brown dwarf): $>0.06M_\odot$, Li燃烧; $>0.013M_\odot$, D燃烧. 光谱型L和T.

零龄主序(zero-age main sequence, ZAMS): 刚到主序, 开始稳定H燃烧.

电离氢区(H II region), 外面是中性氢区(H I region), Str\"omgren半径: 单位体积单位时间复合数$\alpha n_\text{e}n_\text{H}$, $n_\text{e}\simeq n_\text{H}$, 恒星每秒辐射的能电离氢的光子数$N$.

OB星协(association): 一堆OB星(不如星团多)一起诞生, 不能引力束缚.

原行星盘(proplyds): 可能形成行星的盘.
