\chapter{星际介质和恒星形成\\The Interstellar Medium and Star Formation}

星际消光(interstellar extinction):
\begin{equation*}
    m_\lambda=M_\lambda+5\log_{10}d-5+A_\lambda.
\end{equation*}
$I_\lambda/I_{\lambda,0}=e^{-\tau_\lambda}\Rightarrow A_\lambda=m_\lambda-m_{\lambda,0}=-2.5\log_{10}(I_\lambda/I_{\lambda,0})=-2.5\log_{10}(e)\,\tau_\lambda$.

色余(color excess) $E(B-V)=(B-V)_\text{observed}-(B-V)_\text{intrinsic}=A_B-A_V$\footnote{这是老师的定义, 书上的定义和老师是反过来的呜呜呜\dots}.

21cm谱: 基态氢, 自旋---自旋耦合\footnote{什么是耦合? Hamilton算符中同时出现质子自旋算符和电子自旋算符就是耦合.}(超精细结构). 可测原因一: 数密度小, 电子不因碰撞跑到其他能级, 长时间后能跃迁. 可测原因二: 尺度大, 柱密度大(即原子数多), 微小辐射可以大量叠加.

Jeans判据: 孤立云, 云的质量大于Jeans质量, 即云的半径大于Jeans半径, 则塌缩.

林忠四郎\footnote{姓林, 名忠四郎.}轨迹(Hayashi track): 完全对流时流体静力学平衡的临界条件. 要在左边, 不能在右边, 在右边的要收缩. 原恒星演化自由落体阶段就在右边, 但形成原恒星后就只能沿轨迹向下.

补充内容:

柱密度(column density) $N_{d}$ : 数密度对路程积分, 单位截面的粒子数. 由第九章各种定义, $\tau_\lambda=\sigma_\lambda N_{d}$.

示踪体(tracer): 假设比例相同.

ISM加热: 宇宙线(带电粒子), 紫外光电离碳原子等各种原因, SN或星风的激波.

ISM冷却: 红外光子发射.

尘埃来源: 冷恒星包层, SN爆炸和星风, 尘埃生长(coagulation)和吸积(accretion)

Jeans判据(criterion): 维里定理, 动能理想气体动能, 势能第十章$-3/5$倍公式, 假设密度均匀, 得Jeans质量和Jeans长度.

Bonner-Ebert质量: 考虑外压强的判据.

自相似塌缩(homologous collapse): 云自由落体, 无压强外推, 等温, 可得自由落体塌缩时间.

自内向外塌缩(inside-out collapse): 密度高的中心密度增加更快.

碎裂(fragmentation): 不等温, 因为云不透明, 辐射加热云.

双极扩散(ambipolar diffusion): 当中性粒子试图穿越磁力线时, 它们与``冻结''的离子发生碰撞, 中性粒子的运动受到抑制. 然而, 如果由于重力的作用, 中性粒子的运动有一个明确的净方向, 它们仍然倾向于在这个方向上缓慢地移动. 这种缓慢的迁移过程被称为双极扩散.

原恒星演化(protostellar evolution)($1M_\odot$): 
\begin{enumerate}
    \item 自由落体阶段, 透明$\Longrightarrow$等温.
    \item 不透明度增加$\Longrightarrow$绝热坍缩$\Longrightarrow$~$T\nearrow$, 接近流体静力平衡, 原恒星(protostar).
    \item $T\sim1000\text{K}$, 尘埃蒸发, 不透明度下降, $T\nearrow$.
    \item $T\sim2000\text{K}$, 分子瓦解, 吸收能量$\Longrightarrow$核再次坍塌$\Longrightarrow$~$T\nearrow$, 氘点燃.
\end{enumerate}
覆盖在核外的材料不断落在核上, 形成激波并加热核.

主序前演化(pre-main-sequence evolution):
\begin{itemize}
    \item $1M_\odot$\begin{enumerate}
        \item 第一百万年, 完全对流, D燃烧.
        \item $T\nearrow$, 电离, 辐射核.
        \item 核反应(pp和CNO)开始.
        \item CNO产能率高$\Longrightarrow$高温度梯度$\Longrightarrow$对流, 核心膨胀, $T\searrow$, $L\searrow$.
        \item 核反应产能远大于引力势能产能, 恒星稳定.
    \end{enumerate}
    \item $<0.5M_\odot$, C不燃烧.
    \item $<0.072M_\odot$, 无H燃烧, 不可持续, 失败.
\end{itemize}
褐矮星(brown dwarf): $>0.06M_\odot$, Li燃烧; $>0.013M_\odot$, D燃烧. 光谱型L和T.

大质量恒星, 温度高, CNO, 离开林忠四郎线横向演化, 到主序仍然有对流核.

零龄主序(zero-age main sequence, ZAMS): 刚到主序, 开始稳定H燃烧.

初始质量函数(initial mass function): 横轴质量, 纵轴单位质量单位体积(或单位面积)恒星产生数.

电离氢区(H II region), 外面是中性氢区(H I region), Str\"omgren半径: 单位体积单位时间复合数$\alpha n_\text{e}n_\text{H}$, $n_\text{e}\simeq n_\text{H}$, 恒星每秒辐射的能电离氢的光子数$N$.

大质量恒星, 星风, 辐射, 吹散云, 阻止其他恒星形成(大质量恒星形成时间短).

OB星协(association): 一堆OB星(不如星团多)一起诞生, 不能引力束缚.

T Tauri星: 小质量主序前恒星, 光度以数日为尺度大快速不规律动, 光谱特殊, 有P Cygni轮廓: 中心波长处大发射线, 但小于中心波长处小吸收, 原因是有壳膨胀, 但也有T Tauri星有反P Cygni轮廓, 甚至来回切换.

FU Orionis星: 一种T Tauri星, 质量吸积率大增, 内盘比恒星亮, 强星风.

Herbig Ae/Be星: A/B星, 强发射线(所以叫Ae/Be).

Herbig-Haro天体: 两端极长狭窄喷流. 

原行星盘(proplyds): 可能形成行星的盘.
